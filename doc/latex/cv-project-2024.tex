%% Generated by Sphinx.
\def\sphinxdocclass{report}
\documentclass[letterpaper,10pt,english]{sphinxmanual}
\ifdefined\pdfpxdimen
   \let\sphinxpxdimen\pdfpxdimen\else\newdimen\sphinxpxdimen
\fi \sphinxpxdimen=.75bp\relax
\ifdefined\pdfimageresolution
    \pdfimageresolution= \numexpr \dimexpr1in\relax/\sphinxpxdimen\relax
\fi
%% let collapsible pdf bookmarks panel have high depth per default
\PassOptionsToPackage{bookmarksdepth=5}{hyperref}

\PassOptionsToPackage{booktabs}{sphinx}
\PassOptionsToPackage{colorrows}{sphinx}

\PassOptionsToPackage{warn}{textcomp}
\usepackage[utf8]{inputenc}
\ifdefined\DeclareUnicodeCharacter
% support both utf8 and utf8x syntaxes
  \ifdefined\DeclareUnicodeCharacterAsOptional
    \def\sphinxDUC#1{\DeclareUnicodeCharacter{"#1}}
  \else
    \let\sphinxDUC\DeclareUnicodeCharacter
  \fi
  \sphinxDUC{00A0}{\nobreakspace}
  \sphinxDUC{2500}{\sphinxunichar{2500}}
  \sphinxDUC{2502}{\sphinxunichar{2502}}
  \sphinxDUC{2514}{\sphinxunichar{2514}}
  \sphinxDUC{251C}{\sphinxunichar{251C}}
  \sphinxDUC{2572}{\textbackslash}
\fi
\usepackage{cmap}
\usepackage[T1]{fontenc}
\usepackage{amsmath,amssymb,amstext}
\usepackage{babel}



\usepackage{tgtermes}
\usepackage{tgheros}
\renewcommand{\ttdefault}{txtt}



\usepackage[Bjarne]{fncychap}
\usepackage{sphinx}

\fvset{fontsize=auto}
\usepackage{geometry}


% Include hyperref last.
\usepackage{hyperref}
% Fix anchor placement for figures with captions.
\usepackage{hypcap}% it must be loaded after hyperref.
% Set up styles of URL: it should be placed after hyperref.
\urlstyle{same}

\addto\captionsenglish{\renewcommand{\contentsname}{Content:}}

\usepackage{sphinxmessages}
\setcounter{tocdepth}{1}


        \usepackage{lmodern}  % Load Latin Modern fonts
        \renewcommand{\familydefault}{\sfdefault}  % Use sans-serif font as the default
    

\title{CV\sphinxhyphen{}Project\sphinxhyphen{}2024}
\date{Sep 18, 2024}
\release{September 2024}
\author{Marco Bonan}
\newcommand{\sphinxlogo}{\vbox{}}
\renewcommand{\releasename}{Release}
\makeindex
\begin{document}

\ifdefined\shorthandoff
  \ifnum\catcode`\=\string=\active\shorthandoff{=}\fi
  \ifnum\catcode`\"=\active\shorthandoff{"}\fi
\fi

\pagestyle{empty}
\sphinxmaketitle
\pagestyle{plain}
\sphinxtableofcontents
\pagestyle{normal}
\phantomsection\label{\detokenize{index::doc}}


\sphinxAtStartPar
This is the project documentation, its aim is to make it more understandable and provide documentation for the \sphinxtitleref{CameraUtils} module.

\sphinxAtStartPar
Project \sphinxhref{https://github.com/TatoPaato/CV-Project.git}{repository}

\begin{sphinxadmonition}{warning}{Warning:}
\sphinxAtStartPar
Before executing any script make sure to correctly {\hyperref[\detokenize{setup:setup}]{\sphinxcrossref{\DUrole{std,std-ref}{setup your environment}}}}.
\end{sphinxadmonition}

\sphinxstepscope


\chapter{Introduction}
\label{\detokenize{introduction:introduction}}\label{\detokenize{introduction:intro}}\label{\detokenize{introduction::doc}}
\sphinxAtStartPar
The code present in the \sphinxcode{\sphinxupquote{CODE}} folder of the repository can be divided into two groups.
\begin{enumerate}
\sphinxsetlistlabels{\arabic}{enumi}{enumii}{}{.}%
\item {} 
\sphinxAtStartPar
The \sphinxcode{\sphinxupquote{CameraUtils}} module contains the definition of classes, methods, and functions written for the project’s purpose.

\item {} 
\sphinxAtStartPar
The Scripts:
\begin{itemize}
\item {} 
\sphinxAtStartPar
\sphinxcode{\sphinxupquote{configuration.py}} is a Python file that contains user\sphinxhyphen{}definable settings that influence the behavior of scripts when they are executed. This approach was chosen for ease of development, as it simplifies the process compared to using command\sphinxhyphen{}line arguments.

\item {} 
\sphinxAtStartPar
\sphinxcode{\sphinxupquote{Calibration\_INT.py}} is a script that, according to the configuration file:
\begin{itemize}
\item {} 
\sphinxAtStartPar
Gets sample frames from chessboard calibration videos and saves them to file OR simply reads calibration images from file if they are already sampled.

\item {} 
\sphinxAtStartPar
Performs the intrinsic calibration of the cameras, according to the type of camera (normal or wide\sphinxhyphen{}lens).

\item {} 
\sphinxAtStartPar
Saves the found parameters to a \sphinxcode{\sphinxupquote{.pkl}} file for later use.

\item {} 
\sphinxAtStartPar
Shows the effect of camera calibration by undistorting the raw images.

\end{itemize}

\item {} 
\sphinxAtStartPar
\sphinxcode{\sphinxupquote{Calibration\_EXT.py}} is a script that reads the intrinsic parameters from the \sphinxcode{\sphinxupquote{.pkl}} files and launches the \sphinxcode{\sphinxupquote{getCorrespondences()}} method to allow the user to identify key point correspondences between the real world and the images. Once a sufficient number of correspondences is acquired, the extrinsic calibration is performed, as well as a plane\sphinxhyphen{}to\sphinxhyphen{}plane mapping between the image and court floor.

\item {} 
\sphinxAtStartPar
\sphinxcode{\sphinxupquote{PointsHighlighting.py}} is a simple graphical user interface to display all camera views and highlight a selected point on all views, using the plane\sphinxhyphen{}to\sphinxhyphen{}plane mapping found with the previous script.

\item {} 
\sphinxAtStartPar
\sphinxcode{\sphinxupquote{Plotter.py}} is a utility script used to generate plots for the written report.

\end{itemize}

\end{enumerate}

\sphinxstepscope


\chapter{Set up the project}
\label{\detokenize{setup:set-up-the-project}}\label{\detokenize{setup:setup}}\label{\detokenize{setup::doc}}
\sphinxAtStartPar
The project was:
\begin{itemize}
\item {} 
\sphinxAtStartPar
developed and tested on Ubuntu 24.04 LTS.

\item {} 
\sphinxAtStartPar
tested on Ubuntu 22.04.

\item {} 
\sphinxAtStartPar
tested on WSL2 \sphinxhyphen{} Ubuntu.

\end{itemize}

\begin{sphinxadmonition}{note}{Note:}
\sphinxAtStartPar
IN WSL2, in order to display interactive plots make sure a backend such as \sphinxtitleref{tk} is installed, by running:

\begin{sphinxVerbatim}[commandchars=\\\{\}]
\PYG{n}{sudo} \PYG{n}{apt} \PYG{n}{install} \PYG{n}{python3}\PYG{o}{\PYGZhy{}}\PYG{n}{tk}
\end{sphinxVerbatim}
\end{sphinxadmonition}


\section{Virtual environments}
\label{\detokenize{setup:virtual-environments}}
\sphinxAtStartPar
The next steps are adapted from \sphinxhref{https://linuxgenie.net/install-pip-ubuntu-24-04/}{this guide}.

\begin{sphinxadmonition}{warning}{Warning:}
\sphinxAtStartPar
According to the latest changes in Debian release, and therefore Ubuntu 24.04, Python packages are not installed or managed via PIP in the external environment/globally. If you use the pip command to manage the Python packages outside the Python environment, you will get the error: \sphinxstylestrong{externally managed environment}. The best solution is to create a Python environment for the project and manage all the necessary packages/modules there using \sphinxstylestrong{pip}.
\end{sphinxadmonition}

\sphinxAtStartPar
Install the \sphinxcode{\sphinxupquote{venv}} module using \sphinxcode{\sphinxupquote{apt}}:

\begin{sphinxVerbatim}[commandchars=\\\{\}]
\PYG{n}{sudo} \PYG{n}{apt} \PYG{n}{install} \PYG{n}{python3}\PYG{o}{\PYGZhy{}}\PYG{n}{venv}
\end{sphinxVerbatim}

\sphinxAtStartPar
Go to the folder with all the project files and create a virtual environment \sphinxcode{\sphinxupquote{\textless{}ve\_name\textgreater{}}} for Python and activate it using the source command:

\begin{sphinxVerbatim}[commandchars=\\\{\}]
\PYG{n}{python3} \PYG{o}{\PYGZhy{}}\PYG{n}{m} \PYG{n}{venv} \PYG{o}{\PYGZlt{}}\PYG{n}{ve\PYGZus{}name}\PYG{o}{\PYGZgt{}}
\PYG{n}{source} \PYG{o}{\PYGZlt{}}\PYG{n}{ve\PYGZus{}name}\PYG{o}{\PYGZgt{}}\PYG{o}{/}\PYG{n+nb}{bin}\PYG{o}{/}\PYG{n}{activate}
\end{sphinxVerbatim}

\sphinxAtStartPar
You can now install the required packages by running:

\begin{sphinxVerbatim}[commandchars=\\\{\}]
\PYG{n}{pip} \PYG{n}{install} \PYG{o}{\PYGZhy{}}\PYG{n}{r} \PYG{n}{requirements}\PYG{o}{.}\PYG{n}{txt}
\end{sphinxVerbatim}

\begin{sphinxadmonition}{note}{Note:}
\sphinxAtStartPar
In Ubuntu 24.04, both pip and pip3 commands refer to python3\sphinxhyphen{}pip, as Python 2 is no longer officially supported.
\end{sphinxadmonition}

\sphinxAtStartPar
\sphinxstylestrong{pip} can be installed globally by running:

\begin{sphinxVerbatim}[commandchars=\\\{\}]
\PYG{n}{sudo} \PYG{n}{apt} \PYG{n}{install} \PYG{n}{python3}\PYG{o}{\PYGZhy{}}\PYG{n}{pip}
\end{sphinxVerbatim}

\sphinxAtStartPar
If everything is correctly setup, you are ready to launch the scripts.

\sphinxstepscope


\chapter{Configuration}
\label{\detokenize{conf:configuration}}\label{\detokenize{conf::doc}}
\sphinxAtStartPar
This script contains the necessary configurations for the calibration scripts used in the project.
Here there is a brief explanation of the options and parameters that are configurable.


\section{Overview}
\label{\detokenize{conf:overview}}\begin{enumerate}
\sphinxsetlistlabels{\arabic}{enumi}{enumii}{}{.}%
\item {} 
\sphinxAtStartPar
\sphinxstylestrong{Flags}.

\item {} 
\sphinxAtStartPar
\sphinxstylestrong{Settings}: Specifies camera properties, and their positions.

\item {} 
\sphinxAtStartPar
\sphinxstylestrong{Folders and File Paths}: Paths used for video input, calibration frames, and outputs.

\item {} 
\sphinxAtStartPar
\sphinxstylestrong{Court Points and Key Points}: Predefined reference points on the court for calibration correspondences.

\item {} 
\sphinxAtStartPar
\sphinxstylestrong{Constants}.

\end{enumerate}


\subsection{Flags}
\label{\detokenize{conf:flags}}\begin{itemize}
\item {} 
\sphinxAtStartPar
\sphinxtitleref{SAVE\_PARAM}: \sphinxcode{\sphinxupquote{bool}}

\sphinxAtStartPar
Controls whether the camera parameters are saved to a file after calibration.
If \sphinxcode{\sphinxupquote{True}}, parameters are saved to \sphinxtitleref{PARAMETER\_FOLDER}

\item {} 
\sphinxAtStartPar
\sphinxtitleref{SHOW\_TEST\_FRAME}: \sphinxcode{\sphinxupquote{bool}}

\sphinxAtStartPar
If \sphinxcode{\sphinxupquote{True}}, shows before/after intrinsic calibration frames of the camera.

\item {} 
\sphinxAtStartPar
\sphinxtitleref{GET\_SAMPLE}: \sphinxcode{\sphinxupquote{bool}} (default: False)

\sphinxAtStartPar
If \sphinxcode{\sphinxupquote{True}}, scans videos to get sample chessboard frames. Otherwise, uses existing frames stored in \sphinxtitleref{SAMPLE\_FOLDER}.

\item {} 
\sphinxAtStartPar
\sphinxtitleref{READ\_PARAM}: \sphinxcode{\sphinxupquote{bool}}

\sphinxAtStartPar
Whether to read camera parameters from a pickle file.

\item {} 
\sphinxAtStartPar
\sphinxtitleref{CROP}: \sphinxcode{\sphinxupquote{bool}} (default: True)

\sphinxAtStartPar
If \sphinxcode{\sphinxupquote{True}}, crops undistorted images after processing.

\end{itemize}


\subsection{Settings}
\label{\detokenize{conf:settings}}\begin{itemize}
\item {} 
\sphinxAtStartPar
\sphinxtitleref{NUM\_OF\_SAMPLES}: \sphinxcode{\sphinxupquote{int}}

\sphinxAtStartPar
Number of calibration frames to sample.

\item {} 
\sphinxAtStartPar
\sphinxtitleref{FRAME\_SKIP}: \sphinxcode{\sphinxupquote{int}}

\sphinxAtStartPar
Number of frames to skip between each calibration frame extraction.

\item {} 
\sphinxAtStartPar
\sphinxtitleref{SQUARE\_SIZE}: \sphinxcode{\sphinxupquote{float}}

\sphinxAtStartPar
The size of a chessboard square.

\item {} 
\sphinxAtStartPar
\sphinxtitleref{CAMS}: \sphinxcode{\sphinxupquote{dict}}

\sphinxAtStartPar
A dictionary that stores basic information about each camera. Includes:
\begin{itemize}
\item {} 
\sphinxAtStartPar
\sphinxstylestrong{number}: The camera’s number.

\item {} 
\sphinxAtStartPar
\sphinxstylestrong{WIDE}: \sphinxcode{\sphinxupquote{bool}} flag indicating if the camera is a wide\sphinxhyphen{}angle camera.

\item {} 
\sphinxAtStartPar
\sphinxstylestrong{position}: \sphinxcode{\sphinxupquote{list}} of floats that specify the camera’s position in 3D space (in meters).

\end{itemize}

\item {} 
\sphinxAtStartPar
\sphinxtitleref{SCALE}: \sphinxcode{\sphinxupquote{float}} (default: 0.2)

\sphinxAtStartPar
Scaling factor 0 \textless{} SCALE \textless{} 1 for reducing plot resolution to speed up plotting.

\item {} 
\sphinxAtStartPar
\sphinxtitleref{TO\_CAL}: \sphinxcode{\sphinxupquote{list}}

\sphinxAtStartPar
List of cameras to handle, calibrate, and plot during the calibration process.

\end{itemize}

\sphinxAtStartPar
Chessboard Calibration Setup
\textasciicircum{}\textasciicircum{}\textasciicircum{}\textasciicircum{}\textasciicircum{}\textasciicircum{}\textasciicircum{}\textasciicircum{}\textasciicircum{}\textasciicircum{}\textasciicircum{}——\sphinxhyphen{}
\begin{itemize}
\item {} 
\sphinxAtStartPar
\sphinxtitleref{CHESSBOARD\_SIZES}: \sphinxcode{\sphinxupquote{dict}}

\sphinxAtStartPar
Specifies the size of the chessboards used for calibration for each camera. Each entry is a tuple representing the rows and columns of the chessboard (e.g., (5, 7) for 5 rows and 7 columns). Change these values only if you are using different calibration videos.

\end{itemize}

\sphinxAtStartPar
Folders and File Paths
\textasciicircum{}\textasciicircum{}\textasciicircum{}\textasciicircum{}\textasciicircum{}\textasciicircum{}\textasciicircum{}\textasciicircum{}\textasciicircum{}\textasciicircum{}\textasciicircum{}\sphinxhyphen{}
\begin{itemize}
\item {} 
\sphinxAtStartPar
\sphinxtitleref{VIDEO\_FOLDER}: \sphinxcode{\sphinxupquote{str}} (default: “./Video”)

\sphinxAtStartPar
Path to the folder containing game videos.

\item {} 
\sphinxAtStartPar
\sphinxtitleref{FORMAT}: \sphinxcode{\sphinxupquote{str}}

\sphinxAtStartPar
The video format used for calibration and game videos.

\item {} 
\sphinxAtStartPar
\sphinxtitleref{CALIBRATION\_FOLDER}: \sphinxcode{\sphinxupquote{str}} (default: “./Video/Calibration”)

\sphinxAtStartPar
Path to the folder containing calibration videos with chessboards.

\item {} 
\sphinxAtStartPar
\sphinxtitleref{SAMPLE\_FOLDER}: \sphinxcode{\sphinxupquote{str}} (default: “./Cameras/ChessBoardSamples”)

\sphinxAtStartPar
Destination folder for sampled frames from calibration videos.

\item {} 
\sphinxAtStartPar
\sphinxtitleref{SAMPLE\_PATH}: \sphinxcode{\sphinxupquote{str}} (default: “./Video/GAME/Samples/”)

\sphinxAtStartPar
Path to the folder where samples from game videos are stored.

\item {} 
\sphinxAtStartPar
\sphinxtitleref{UNDISTORTED\_SAMPLES}: \sphinxcode{\sphinxupquote{str}} (default: “./Cameras/UndistortedSamples”)

\sphinxAtStartPar
Path to store undistorted image samples after calibration.

\item {} 
\sphinxAtStartPar
\sphinxtitleref{PARAMETER\_FOLDER}: \sphinxcode{\sphinxupquote{str}} (default: “./Cameras/Parameters”)

\sphinxAtStartPar
Folder to store the camera parameters after calibration.

\item {} 
\sphinxAtStartPar
\sphinxtitleref{ERROR\_FOLDER}: \sphinxcode{\sphinxupquote{str}} (default: “./Cameras/Errors”)

\sphinxAtStartPar
Folder to store the calibration errors.

\item {} 
\sphinxAtStartPar
\sphinxtitleref{PLOT\_FOLDER}: \sphinxcode{\sphinxupquote{str}} (default: “./Plots”)

\sphinxAtStartPar
Folder to store plots.

\end{itemize}


\subsection{Court and Plotting Settings}
\label{\detokenize{conf:court-and-plotting-settings}}\begin{itemize}
\item {} 
\sphinxAtStartPar
\sphinxtitleref{SHOW\_COURT}: \sphinxcode{\sphinxupquote{bool}} (default: True)

\sphinxAtStartPar
Whether to display the court in plots.

\item {} 
\sphinxAtStartPar
\sphinxtitleref{COURT\_IMG\_LR}, \sphinxtitleref{COURT\_IMG\_MR}, \sphinxtitleref{COURT\_IMG\_XL}: \sphinxcode{\sphinxupquote{str}}

\sphinxAtStartPar
Paths to court images of varying resolutions (low, medium, and extra\sphinxhyphen{}large) for different plotting scenarios.

\item {} 
\sphinxAtStartPar
\sphinxtitleref{COURT\_IMG\_QUALITY}: \sphinxcode{\sphinxupquote{int}} (default: 1)

\sphinxAtStartPar
Controls the rendering quality of the court image. Lower values correspond to higher quality but slower rendering.

\end{itemize}


\subsection{Court Points and Key Points}
\label{\detokenize{conf:court-points-and-key-points}}\begin{itemize}
\item {} 
\sphinxAtStartPar
\sphinxtitleref{COURT\_POINTS}: \sphinxcode{\sphinxupquote{np.array}}

\sphinxAtStartPar
A numpy array defining key reference points on the court for calibration purposes. Points are defined for both volleyball and basketball courts, and they correspond to real\sphinxhyphen{}world 3D coordinates (in meters). These points are used for correspondence matching in the calibration process.

\end{itemize}


\subsection{Calibration Constants}
\label{\detokenize{conf:calibration-constants}}\begin{itemize}
\item {} 
\sphinxAtStartPar
\sphinxtitleref{MIN\_POINTS}: \sphinxcode{\sphinxupquote{int}} (default: 6)

\sphinxAtStartPar
The minimum number of points required for external calibration, specifically when using planar 3D points. This ensures compatibility with the calibration system’s requirement of using at least 6 points for accurate pose estimation and homography decomposition.

\begin{sphinxadmonition}{note}{Note:}
\sphinxAtStartPar
On the minimum number of points:
\begin{itemize}
\item {} 
\sphinxAtStartPar
\sphinxtitleref{cv2.solvePnP}: Needs at least 6 points for non\sphinxhyphen{}planar object points and 4 points for planar objects.

\item {} 
\sphinxAtStartPar
\sphinxtitleref{cv2.findHomography}: Requires at least 4 points but uses a minimum of 6 in this code to improve stability and consistency.

\end{itemize}
\end{sphinxadmonition}

\end{itemize}

\sphinxstepscope


\chapter{CameraUtils}
\label{\detokenize{CameraUtils:camerautils}}\label{\detokenize{CameraUtils:docs}}\label{\detokenize{CameraUtils::doc}}

\section{Classes}
\label{\detokenize{CameraUtils:classes}}\index{Camera (class in CameraUtils)@\spxentry{Camera}\spxextra{class in CameraUtils}}

\begin{fulllineitems}
\phantomsection\label{\detokenize{CameraUtils:CameraUtils.Camera}}
\pysigstartsignatures
\pysiglinewithargsret{\sphinxbfcode{\sphinxupquote{class\DUrole{w}{ }}}\sphinxcode{\sphinxupquote{CameraUtils.}}\sphinxbfcode{\sphinxupquote{Camera}}}{\sphinxparam{\DUrole{n}{camera\_number}}\sphinxparamcomma \sphinxparam{\DUrole{n}{approximate\_position}}\sphinxparamcomma \sphinxparam{\DUrole{n}{WIDE\_LENS}\DUrole{o}{=}\DUrole{default_value}{False}}}{}
\pysigstopsignatures
\sphinxAtStartPar
A class used to represent the calibration process of a camera using chessboard images.
\index{camera\_number (CameraUtils.Camera attribute)@\spxentry{camera\_number}\spxextra{CameraUtils.Camera attribute}}

\begin{fulllineitems}
\phantomsection\label{\detokenize{CameraUtils:CameraUtils.Camera.camera_number}}
\pysigstartsignatures
\pysigline{\sphinxbfcode{\sphinxupquote{camera\_number}}}
\pysigstopsignatures
\sphinxAtStartPar
The identifier number for the camera being calibrated.
\begin{quote}\begin{description}
\sphinxlineitem{Type}
\sphinxAtStartPar
int

\end{description}\end{quote}

\end{fulllineitems}

\index{FISHEYE (CameraUtils.Camera attribute)@\spxentry{FISHEYE}\spxextra{CameraUtils.Camera attribute}}

\begin{fulllineitems}
\phantomsection\label{\detokenize{CameraUtils:CameraUtils.Camera.FISHEYE}}
\pysigstartsignatures
\pysigline{\sphinxbfcode{\sphinxupquote{FISHEYE}}}
\pysigstopsignatures
\sphinxAtStartPar
A flag indicating if the camera is a WideLens.
\begin{quote}\begin{description}
\sphinxlineitem{Type}
\sphinxAtStartPar
bool

\end{description}\end{quote}

\end{fulllineitems}

\index{chessboard\_size (CameraUtils.Camera attribute)@\spxentry{chessboard\_size}\spxextra{CameraUtils.Camera attribute}}

\begin{fulllineitems}
\phantomsection\label{\detokenize{CameraUtils:CameraUtils.Camera.chessboard_size}}
\pysigstartsignatures
\pysigline{\sphinxbfcode{\sphinxupquote{chessboard\_size}}}
\pysigstopsignatures
\sphinxAtStartPar
The number of inner corners per chessboard row and column (e.g., (7, 6)).
\begin{quote}\begin{description}
\sphinxlineitem{Type}
\sphinxAtStartPar
tuple or None

\end{description}\end{quote}

\end{fulllineitems}

\index{obj\_points (CameraUtils.Camera attribute)@\spxentry{obj\_points}\spxextra{CameraUtils.Camera attribute}}

\begin{fulllineitems}
\phantomsection\label{\detokenize{CameraUtils:CameraUtils.Camera.obj_points}}
\pysigstartsignatures
\pysigline{\sphinxbfcode{\sphinxupquote{obj\_points}}}
\pysigstopsignatures
\sphinxAtStartPar
The 3D points in the real\sphinxhyphen{}world space.
\begin{quote}\begin{description}
\sphinxlineitem{Type}
\sphinxAtStartPar
list

\end{description}\end{quote}

\end{fulllineitems}

\index{img\_points (CameraUtils.Camera attribute)@\spxentry{img\_points}\spxextra{CameraUtils.Camera attribute}}

\begin{fulllineitems}
\phantomsection\label{\detokenize{CameraUtils:CameraUtils.Camera.img_points}}
\pysigstartsignatures
\pysigline{\sphinxbfcode{\sphinxupquote{img\_points}}}
\pysigstopsignatures
\sphinxAtStartPar
The 2D points in the image plane.
\begin{quote}\begin{description}
\sphinxlineitem{Type}
\sphinxAtStartPar
list

\end{description}\end{quote}

\end{fulllineitems}

\index{mtx (CameraUtils.Camera attribute)@\spxentry{mtx}\spxextra{CameraUtils.Camera attribute}}

\begin{fulllineitems}
\phantomsection\label{\detokenize{CameraUtils:CameraUtils.Camera.mtx}}
\pysigstartsignatures
\pysigline{\sphinxbfcode{\sphinxupquote{mtx}}}
\pysigstopsignatures
\sphinxAtStartPar
The camera matrix.
\begin{quote}\begin{description}
\sphinxlineitem{Type}
\sphinxAtStartPar
numpy.ndarray or None

\end{description}\end{quote}

\end{fulllineitems}

\index{dist (CameraUtils.Camera attribute)@\spxentry{dist}\spxextra{CameraUtils.Camera attribute}}

\begin{fulllineitems}
\phantomsection\label{\detokenize{CameraUtils:CameraUtils.Camera.dist}}
\pysigstartsignatures
\pysigline{\sphinxbfcode{\sphinxupquote{dist}}}
\pysigstopsignatures
\sphinxAtStartPar
The distortion coefficients.
\begin{quote}\begin{description}
\sphinxlineitem{Type}
\sphinxAtStartPar
numpy.ndarray or None

\end{description}\end{quote}

\end{fulllineitems}

\index{rvecs (CameraUtils.Camera attribute)@\spxentry{rvecs}\spxextra{CameraUtils.Camera attribute}}

\begin{fulllineitems}
\phantomsection\label{\detokenize{CameraUtils:CameraUtils.Camera.rvecs}}
\pysigstartsignatures
\pysigline{\sphinxbfcode{\sphinxupquote{rvecs}}}
\pysigstopsignatures
\sphinxAtStartPar
The rotation vectors estimated for each pattern view.
\begin{quote}\begin{description}
\sphinxlineitem{Type}
\sphinxAtStartPar
list or None

\end{description}\end{quote}

\end{fulllineitems}

\index{tvecs (CameraUtils.Camera attribute)@\spxentry{tvecs}\spxextra{CameraUtils.Camera attribute}}

\begin{fulllineitems}
\phantomsection\label{\detokenize{CameraUtils:CameraUtils.Camera.tvecs}}
\pysigstartsignatures
\pysigline{\sphinxbfcode{\sphinxupquote{tvecs}}}
\pysigstopsignatures
\sphinxAtStartPar
The translation vectors estimated for each pattern view.
\begin{quote}\begin{description}
\sphinxlineitem{Type}
\sphinxAtStartPar
list or None

\end{description}\end{quote}

\end{fulllineitems}

\index{new\_mtx (CameraUtils.Camera attribute)@\spxentry{new\_mtx}\spxextra{CameraUtils.Camera attribute}}

\begin{fulllineitems}
\phantomsection\label{\detokenize{CameraUtils:CameraUtils.Camera.new_mtx}}
\pysigstartsignatures
\pysigline{\sphinxbfcode{\sphinxupquote{new\_mtx}}}
\pysigstopsignatures
\sphinxAtStartPar
The refined camera matrix used for undistortion.
\begin{quote}\begin{description}
\sphinxlineitem{Type}
\sphinxAtStartPar
numpy.ndarray or None

\end{description}\end{quote}

\end{fulllineitems}

\index{INT\_CAL (CameraUtils.Camera attribute)@\spxentry{INT\_CAL}\spxextra{CameraUtils.Camera attribute}}

\begin{fulllineitems}
\phantomsection\label{\detokenize{CameraUtils:CameraUtils.Camera.INT_CAL}}
\pysigstartsignatures
\pysigline{\sphinxbfcode{\sphinxupquote{INT\_CAL}}}
\pysigstopsignatures
\sphinxAtStartPar
A flag indicating whether the camera has intrinsic calibration.
\begin{quote}\begin{description}
\sphinxlineitem{Type}
\sphinxAtStartPar
bool

\end{description}\end{quote}

\end{fulllineitems}

\index{EXT\_CAL (CameraUtils.Camera attribute)@\spxentry{EXT\_CAL}\spxextra{CameraUtils.Camera attribute}}

\begin{fulllineitems}
\phantomsection\label{\detokenize{CameraUtils:CameraUtils.Camera.EXT_CAL}}
\pysigstartsignatures
\pysigline{\sphinxbfcode{\sphinxupquote{EXT\_CAL}}}
\pysigstopsignatures
\sphinxAtStartPar
A flag indicating whether the camera has extrinsic calibration.
\begin{quote}\begin{description}
\sphinxlineitem{Type}
\sphinxAtStartPar
bool

\end{description}\end{quote}

\end{fulllineitems}

\index{img\_size (CameraUtils.Camera attribute)@\spxentry{img\_size}\spxextra{CameraUtils.Camera attribute}}

\begin{fulllineitems}
\phantomsection\label{\detokenize{CameraUtils:CameraUtils.Camera.img_size}}
\pysigstartsignatures
\pysigline{\sphinxbfcode{\sphinxupquote{img\_size}}}
\pysigstopsignatures
\sphinxAtStartPar
The size of the images used for calibration (h, w, channels).
\begin{quote}\begin{description}
\sphinxlineitem{Type}
\sphinxAtStartPar
list

\end{description}\end{quote}

\end{fulllineitems}

\index{roi (CameraUtils.Camera attribute)@\spxentry{roi}\spxextra{CameraUtils.Camera attribute}}

\begin{fulllineitems}
\phantomsection\label{\detokenize{CameraUtils:CameraUtils.Camera.roi}}
\pysigstartsignatures
\pysigline{\sphinxbfcode{\sphinxupquote{roi}}}
\pysigstopsignatures
\sphinxAtStartPar
The region of interest for the undistorted images.
\begin{quote}\begin{description}
\sphinxlineitem{Type}
\sphinxAtStartPar
tuple or None

\end{description}\end{quote}

\end{fulllineitems}

\index{corr\_world\_points (CameraUtils.Camera attribute)@\spxentry{corr\_world\_points}\spxextra{CameraUtils.Camera attribute}}

\begin{fulllineitems}
\phantomsection\label{\detokenize{CameraUtils:CameraUtils.Camera.corr_world_points}}
\pysigstartsignatures
\pysigline{\sphinxbfcode{\sphinxupquote{corr\_world\_points}}}
\pysigstopsignatures
\sphinxAtStartPar
A numpy array of points in 3D space, in the court reference frame.
\begin{quote}\begin{description}
\sphinxlineitem{Type}
\sphinxAtStartPar
numpy.ndarray or None

\end{description}\end{quote}

\end{fulllineitems}

\index{corr\_image\_points (CameraUtils.Camera attribute)@\spxentry{corr\_image\_points}\spxextra{CameraUtils.Camera attribute}}

\begin{fulllineitems}
\phantomsection\label{\detokenize{CameraUtils:CameraUtils.Camera.corr_image_points}}
\pysigstartsignatures
\pysigline{\sphinxbfcode{\sphinxupquote{corr\_image\_points}}}
\pysigstopsignatures
\sphinxAtStartPar
A numpy array of points in the 2D image coordinates corresponding to
the real\sphinxhyphen{}world points (corr\_world\_points).
\begin{quote}\begin{description}
\sphinxlineitem{Type}
\sphinxAtStartPar
numpy.ndarray or None

\end{description}\end{quote}

\end{fulllineitems}

\index{ext\_mtx (CameraUtils.Camera attribute)@\spxentry{ext\_mtx}\spxextra{CameraUtils.Camera attribute}}

\begin{fulllineitems}
\phantomsection\label{\detokenize{CameraUtils:CameraUtils.Camera.ext_mtx}}
\pysigstartsignatures
\pysigline{\sphinxbfcode{\sphinxupquote{ext\_mtx}}}
\pysigstopsignatures
\sphinxAtStartPar
The matrix representing the camera\sphinxhyphen{}to\sphinxhyphen{}world transformation.
\begin{quote}\begin{description}
\sphinxlineitem{Type}
\sphinxAtStartPar
numpy.ndarray or None

\end{description}\end{quote}

\end{fulllineitems}

\index{H\_mtx (CameraUtils.Camera attribute)@\spxentry{H\_mtx}\spxextra{CameraUtils.Camera attribute}}

\begin{fulllineitems}
\phantomsection\label{\detokenize{CameraUtils:CameraUtils.Camera.H_mtx}}
\pysigstartsignatures
\pysigline{\sphinxbfcode{\sphinxupquote{H\_mtx}}}
\pysigstopsignatures
\sphinxAtStartPar
The homography matrix between the court and image plane.
\begin{quote}\begin{description}
\sphinxlineitem{Type}
\sphinxAtStartPar
numpy.ndarray

\end{description}\end{quote}

\end{fulllineitems}

\index{rep\_err (CameraUtils.Camera attribute)@\spxentry{rep\_err}\spxextra{CameraUtils.Camera attribute}}

\begin{fulllineitems}
\phantomsection\label{\detokenize{CameraUtils:CameraUtils.Camera.rep_err}}
\pysigstartsignatures
\pysigline{\sphinxbfcode{\sphinxupquote{rep\_err}}}
\pysigstopsignatures
\sphinxAtStartPar
The overall reprojection error after intrinsic calibration, representing the quality of the calibration.
\begin{quote}\begin{description}
\sphinxlineitem{Type}
\sphinxAtStartPar
float or None

\end{description}\end{quote}

\end{fulllineitems}

\index{pos\_err (CameraUtils.Camera attribute)@\spxentry{pos\_err}\spxextra{CameraUtils.Camera attribute}}

\begin{fulllineitems}
\phantomsection\label{\detokenize{CameraUtils:CameraUtils.Camera.pos_err}}
\pysigstartsignatures
\pysigline{\sphinxbfcode{\sphinxupquote{pos\_err}}}
\pysigstopsignatures
\sphinxAtStartPar
A vector representing the positional error in 3D space after extrinsic calibration, initialized to zeros.
\begin{quote}\begin{description}
\sphinxlineitem{Type}
\sphinxAtStartPar
numpy.ndarray

\end{description}\end{quote}

\end{fulllineitems}

\index{\_\_init\_\_() (CameraUtils.Camera method)@\spxentry{\_\_init\_\_()}\spxextra{CameraUtils.Camera method}}

\begin{fulllineitems}
\phantomsection\label{\detokenize{CameraUtils:CameraUtils.Camera.__init__}}
\pysigstartsignatures
\pysiglinewithargsret{\sphinxbfcode{\sphinxupquote{\_\_init\_\_}}}{\sphinxparam{\DUrole{n}{camera\_number}}\sphinxparamcomma \sphinxparam{\DUrole{n}{approximate\_position}}\sphinxparamcomma \sphinxparam{\DUrole{n}{WIDE\_LENS}\DUrole{o}{=}\DUrole{default_value}{False}}}{}
\pysigstopsignatures
\sphinxAtStartPar
Initializes the Camera object with a specific camera number, an approximate position in the real world,
and whether or not it is WideLens. Sets all other attributes to default values.

\end{fulllineitems}

\index{GetSampleFrames() (CameraUtils.Camera method)@\spxentry{GetSampleFrames()}\spxextra{CameraUtils.Camera method}}

\begin{fulllineitems}
\phantomsection\label{\detokenize{CameraUtils:CameraUtils.Camera.GetSampleFrames}}
\pysigstartsignatures
\pysiglinewithargsret{\sphinxbfcode{\sphinxupquote{GetSampleFrames}}}{\sphinxparam{\DUrole{n}{video\_dir}}\sphinxparamcomma \sphinxparam{\DUrole{n}{frame\_skip}}\sphinxparamcomma \sphinxparam{\DUrole{n}{out\_dir}}}{}
\pysigstopsignatures
\sphinxAtStartPar
Extracts sample frames from a video to use for camera calibration.

\end{fulllineitems}

\index{GetCornersFromSamples() (CameraUtils.Camera method)@\spxentry{GetCornersFromSamples()}\spxextra{CameraUtils.Camera method}}

\begin{fulllineitems}
\phantomsection\label{\detokenize{CameraUtils:CameraUtils.Camera.GetCornersFromSamples}}
\pysigstartsignatures
\pysiglinewithargsret{\sphinxbfcode{\sphinxupquote{GetCornersFromSamples}}}{\sphinxparam{\DUrole{n}{sample\_folder}}}{}
\pysigstopsignatures
\sphinxAtStartPar
Reads existing images from a folder of existing samples to use for camera calibration.

\end{fulllineitems}

\index{CalibrateCamera() (CameraUtils.Camera method)@\spxentry{CalibrateCamera()}\spxextra{CameraUtils.Camera method}}

\begin{fulllineitems}
\phantomsection\label{\detokenize{CameraUtils:CameraUtils.Camera.CalibrateCamera}}
\pysigstartsignatures
\pysiglinewithargsret{\sphinxbfcode{\sphinxupquote{CalibrateCamera}}}{}{}
\pysigstopsignatures
\sphinxAtStartPar
Performs intrinsic calibration of the camera and sets the status to calibrated.

\end{fulllineitems}

\index{SaveParameters() (CameraUtils.Camera method)@\spxentry{SaveParameters()}\spxextra{CameraUtils.Camera method}}

\begin{fulllineitems}
\phantomsection\label{\detokenize{CameraUtils:CameraUtils.Camera.SaveParameters}}
\pysigstartsignatures
\pysiglinewithargsret{\sphinxbfcode{\sphinxupquote{SaveParameters}}}{\sphinxparam{\DUrole{n}{save\_dir}}}{}
\pysigstopsignatures
\sphinxAtStartPar
Saves camera parameters to a .yaml file.

\end{fulllineitems}

\index{ReadParameters() (CameraUtils.Camera method)@\spxentry{ReadParameters()}\spxextra{CameraUtils.Camera method}}

\begin{fulllineitems}
\phantomsection\label{\detokenize{CameraUtils:CameraUtils.Camera.ReadParameters}}
\pysigstartsignatures
\pysiglinewithargsret{\sphinxbfcode{\sphinxupquote{ReadParameters}}}{\sphinxparam{\DUrole{n}{param\_dir}}}{}
\pysigstopsignatures
\sphinxAtStartPar
Reads camera parameters from a .yaml file.

\end{fulllineitems}

\index{NewOptimalCameraMatrix() (CameraUtils.Camera method)@\spxentry{NewOptimalCameraMatrix()}\spxextra{CameraUtils.Camera method}}

\begin{fulllineitems}
\phantomsection\label{\detokenize{CameraUtils:CameraUtils.Camera.NewOptimalCameraMatrix}}
\pysigstartsignatures
\pysiglinewithargsret{\sphinxbfcode{\sphinxupquote{NewOptimalCameraMatrix}}}{}{}
\pysigstopsignatures
\sphinxAtStartPar
Refines the camera matrix for the undistorsion.

\end{fulllineitems}

\index{TestUndistorsion() (CameraUtils.Camera method)@\spxentry{TestUndistorsion()}\spxextra{CameraUtils.Camera method}}

\begin{fulllineitems}
\phantomsection\label{\detokenize{CameraUtils:CameraUtils.Camera.TestUndistorsion}}
\pysigstartsignatures
\pysiglinewithargsret{\sphinxbfcode{\sphinxupquote{TestUndistorsion}}}{}{}
\pysigstopsignatures
\sphinxAtStartPar
Applies undistorsion to the image and shows the effect on an example.

\end{fulllineitems}

\index{GetCorrespondences() (CameraUtils.Camera method)@\spxentry{GetCorrespondences()}\spxextra{CameraUtils.Camera method}}

\begin{fulllineitems}
\phantomsection\label{\detokenize{CameraUtils:CameraUtils.Camera.GetCorrespondences}}
\pysigstartsignatures
\pysiglinewithargsret{\sphinxbfcode{\sphinxupquote{GetCorrespondences}}}{\sphinxparam{\DUrole{n}{world\_points}}\sphinxparamcomma \sphinxparam{\DUrole{n}{image\_path}}}{}
\pysigstopsignatures
\sphinxAtStartPar
Obtain correspondences between world points and image points through
a manual user interface.

\end{fulllineitems}

\index{AddManualCorrespondences() (CameraUtils.Camera method)@\spxentry{AddManualCorrespondences()}\spxextra{CameraUtils.Camera method}}

\begin{fulllineitems}
\phantomsection\label{\detokenize{CameraUtils:CameraUtils.Camera.AddManualCorrespondences}}
\pysigstartsignatures
\pysiglinewithargsret{\sphinxbfcode{\sphinxupquote{AddManualCorrespondences}}}{\sphinxparam{\DUrole{n}{world\_points}}\sphinxparamcomma \sphinxparam{\DUrole{n}{image\_points}}}{}
\pysigstopsignatures
\sphinxAtStartPar
Manually add corresponding points to camera attributes.

\end{fulllineitems}

\index{ExtCalibration() (CameraUtils.Camera method)@\spxentry{ExtCalibration()}\spxextra{CameraUtils.Camera method}}

\begin{fulllineitems}
\phantomsection\label{\detokenize{CameraUtils:CameraUtils.Camera.ExtCalibration}}
\pysigstartsignatures
\pysiglinewithargsret{\sphinxbfcode{\sphinxupquote{ExtCalibration}}}{}{}
\pysigstopsignatures
\sphinxAtStartPar
Perform external camera calibration to compute the transformation matrices
between the world and camera coordinates, also compute the homography
matrix that map the image and court planes.

\end{fulllineitems}

\index{PosError() (CameraUtils.Camera method)@\spxentry{PosError()}\spxextra{CameraUtils.Camera method}}

\begin{fulllineitems}
\phantomsection\label{\detokenize{CameraUtils:CameraUtils.Camera.PosError}}
\pysigstartsignatures
\pysiglinewithargsret{\sphinxbfcode{\sphinxupquote{PosError}}}{}{}
\pysigstopsignatures
\sphinxAtStartPar
Calculate and display the positional error between the approximate
and estimated positions

\end{fulllineitems}

\index{SaveErrors() (CameraUtils.Camera method)@\spxentry{SaveErrors()}\spxextra{CameraUtils.Camera method}}

\begin{fulllineitems}
\phantomsection\label{\detokenize{CameraUtils:CameraUtils.Camera.SaveErrors}}
\pysigstartsignatures
\pysiglinewithargsret{\sphinxbfcode{\sphinxupquote{SaveErrors}}}{\sphinxparam{\DUrole{n}{\textasciigrave{}camera\_list\textasciigrave{}}}\sphinxparamcomma \sphinxparam{\DUrole{n}{path}}}{}
\pysigstopsignatures
\sphinxAtStartPar
\sphinxstyleemphasis{This is a class method.}
Save the positional and re\sphinxhyphen{}projection errors of a list of cameras to a CSV file

\end{fulllineitems}

\index{PlotCamera() (CameraUtils.Camera method)@\spxentry{PlotCamera()}\spxextra{CameraUtils.Camera method}}

\begin{fulllineitems}
\phantomsection\label{\detokenize{CameraUtils:CameraUtils.Camera.PlotCamera}}
\pysigstartsignatures
\pysiglinewithargsret{\sphinxbfcode{\sphinxupquote{PlotCamera}}}{\sphinxparam{\DUrole{n}{ax}}}{}
\pysigstopsignatures
\sphinxAtStartPar
Plot the camera’s position and direction on a 3D axis. {\color{red}\bfseries{}*}This method is intended to be called by \sphinxtitleref{PlotMultipleCameras()} class method

\end{fulllineitems}

\index{PlotMultipleCameras() (CameraUtils.Camera method)@\spxentry{PlotMultipleCameras()}\spxextra{CameraUtils.Camera method}}

\begin{fulllineitems}
\phantomsection\label{\detokenize{CameraUtils:CameraUtils.Camera.PlotMultipleCameras}}
\pysigstartsignatures
\pysiglinewithargsret{\sphinxbfcode{\sphinxupquote{PlotMultipleCameras}}}{\sphinxparam{\DUrole{n}{camera\_list}}}{}
\pysigstopsignatures
\sphinxAtStartPar
\sphinxstyleemphasis{This is a class method}
Plot the positions and orientations of multiple cameras on a 3D plot.

\end{fulllineitems}

\index{PlotCamera2D() (CameraUtils.Camera method)@\spxentry{PlotCamera2D()}\spxextra{CameraUtils.Camera method}}

\begin{fulllineitems}
\phantomsection\label{\detokenize{CameraUtils:CameraUtils.Camera.PlotCamera2D}}
\pysigstartsignatures
\pysiglinewithargsret{\sphinxbfcode{\sphinxupquote{PlotCamera2D}}}{\sphinxparam{\DUrole{n}{ax}}}{}
\pysigstopsignatures
\sphinxAtStartPar
Plot the camera’s position and direction on an axis. {\color{red}\bfseries{}*}This method is intended to be called by \sphinxtitleref{PlotMultipleCameras2D()} class method.

\end{fulllineitems}

\index{PlotMultipleCameras2D() (CameraUtils.Camera method)@\spxentry{PlotMultipleCameras2D()}\spxextra{CameraUtils.Camera method}}

\begin{fulllineitems}
\phantomsection\label{\detokenize{CameraUtils:CameraUtils.Camera.PlotMultipleCameras2D}}
\pysigstartsignatures
\pysiglinewithargsret{\sphinxbfcode{\sphinxupquote{PlotMultipleCameras2D}}}{\sphinxparam{\DUrole{n}{camera\_list}}}{}
\pysigstopsignatures
\sphinxAtStartPar
\sphinxstyleemphasis{This is a class method}
Plot the positions and orientations of multiple cameras on a 2D court image.

\end{fulllineitems}

\index{PrintAttributes() (CameraUtils.Camera method)@\spxentry{PrintAttributes()}\spxextra{CameraUtils.Camera method}}

\begin{fulllineitems}
\phantomsection\label{\detokenize{CameraUtils:CameraUtils.Camera.PrintAttributes}}
\pysigstartsignatures
\pysiglinewithargsret{\sphinxbfcode{\sphinxupquote{PrintAttributes}}}{\sphinxparam{\DUrole{n}{skip\_attributes}}}{}
\pysigstopsignatures
\sphinxAtStartPar
Prints all the attributes of the CameraInfo instance, except for those in the skip\_attributes list.
Method useful for quick attributes checks/debugging

\end{fulllineitems}

\index{FindHomography() (CameraUtils.Camera method)@\spxentry{FindHomography()}\spxextra{CameraUtils.Camera method}}

\begin{fulllineitems}
\phantomsection\label{\detokenize{CameraUtils:CameraUtils.Camera.FindHomography}}
\pysigstartsignatures
\pysiglinewithargsret{\sphinxbfcode{\sphinxupquote{FindHomography}}}{}{}
\pysigstopsignatures
\sphinxAtStartPar
Compute the homography matrix from world coordinates (court plane) to image coordinates.

\end{fulllineitems}

\index{Court2Image() (CameraUtils.Camera method)@\spxentry{Court2Image()}\spxextra{CameraUtils.Camera method}}

\begin{fulllineitems}
\phantomsection\label{\detokenize{CameraUtils:CameraUtils.Camera.Court2Image}}
\pysigstartsignatures
\pysiglinewithargsret{\sphinxbfcode{\sphinxupquote{Court2Image}}}{\sphinxparam{\DUrole{n}{coords}}}{}
\pysigstopsignatures
\sphinxAtStartPar
Convert court coordinates to image coordinates using the homography matrix

\end{fulllineitems}

\index{Image2Court() (CameraUtils.Camera method)@\spxentry{Image2Court()}\spxextra{CameraUtils.Camera method}}

\begin{fulllineitems}
\phantomsection\label{\detokenize{CameraUtils:CameraUtils.Camera.Image2Court}}
\pysigstartsignatures
\pysiglinewithargsret{\sphinxbfcode{\sphinxupquote{Image2Court}}}{\sphinxparam{\DUrole{n}{coords}}}{}
\pysigstopsignatures
\sphinxAtStartPar
Convert image coordinates to court coordinates using the homography matrix.

\end{fulllineitems}

\index{AddManualCorrespondences() (CameraUtils.Camera method)@\spxentry{AddManualCorrespondences()}\spxextra{CameraUtils.Camera method}}

\begin{fulllineitems}
\phantomsection\label{\detokenize{CameraUtils:id0}}
\pysigstartsignatures
\pysiglinewithargsret{\sphinxbfcode{\sphinxupquote{AddManualCorrespondences}}}{\sphinxparam{\DUrole{n}{world\_points}}\sphinxparamcomma \sphinxparam{\DUrole{n}{image\_points}}}{}
\pysigstopsignatures
\sphinxAtStartPar
Manually add corresponding points to camera attributes after performing type and NaN checks.
\begin{quote}\begin{description}
\sphinxlineitem{Parameters}\begin{itemize}
\item {} 
\sphinxAtStartPar
\sphinxstyleliteralstrong{\sphinxupquote{world\_points}} (\sphinxstyleliteralemphasis{\sphinxupquote{numpy.ndarray}}) \textendash{} Point coordinates (x, y, z) expressed in a numpy array with shape (n, 3),
where \sphinxtitleref{n} is the number of points. Example: \sphinxtitleref{{[}{[}x, y, z{]}, …{]}}

\item {} 
\sphinxAtStartPar
\sphinxstyleliteralstrong{\sphinxupquote{image\_points}} (\sphinxstyleliteralemphasis{\sphinxupquote{numpy.ndarray}}) \textendash{} Point coordinates (u, v) expressed in a numpy array with shape (n, 2),
where \sphinxtitleref{n} is the number of points. Example: \sphinxtitleref{{[}{[}u, v{]}, …{]}}

\end{itemize}

\sphinxlineitem{Returns}
\sphinxAtStartPar
None

\sphinxlineitem{Return type}
\sphinxAtStartPar
None

\sphinxlineitem{Notes}\begin{itemize}
\item {} 
\sphinxAtStartPar
This method verifies that both input arrays are of type \sphinxtitleref{numpy.ndarray}.

\item {} 
\sphinxAtStartPar
The input arrays are checked to have the correct shapes: \sphinxtitleref{(n, 3)} for \sphinxtitleref{world\_points} and \sphinxtitleref{(n, 2)} for \sphinxtitleref{image\_points}.

\item {} 
\sphinxAtStartPar
NaN values within the input arrays are filtered out before the points are stored.

\item {} 
\sphinxAtStartPar
If any of these checks fail, appropriate error messages will be printed.

\end{itemize}

\end{description}\end{quote}

\end{fulllineitems}

\index{CalibrateCamera() (CameraUtils.Camera method)@\spxentry{CalibrateCamera()}\spxextra{CameraUtils.Camera method}}

\begin{fulllineitems}
\phantomsection\label{\detokenize{CameraUtils:id5}}
\pysigstartsignatures
\pysiglinewithargsret{\sphinxbfcode{\sphinxupquote{CalibrateCamera}}}{}{}
\pysigstopsignatures
\sphinxAtStartPar
Calibrate the camera and set the status to calibrated.

\sphinxAtStartPar
This method performs the camera calibration process using the collected
object points and image points. It computes the camera matrix and
distortion coefficients. The calibration process is different for
fisheye lenses and standard lenses.
\begin{quote}\begin{description}
\sphinxlineitem{Raises}
\sphinxAtStartPar
\sphinxstyleliteralstrong{\sphinxupquote{RuntimeError}} \textendash{} If there is an error during the calibration process.

\end{description}\end{quote}
\subsubsection*{Notes}
\begin{itemize}
\item {} 
\sphinxAtStartPar
For fisheye lenses, it uses \sphinxtitleref{cv2.fisheye.calibrate} and sets the fisheye

\end{itemize}

\sphinxAtStartPar
calibration flags.
\sphinxhyphen{} For standard lenses, it uses \sphinxtitleref{cv2.calibrateCamera}.
\sphinxhyphen{} Sets \sphinxtitleref{self.INT\_CAL} to True if the calibration is successful.
\sphinxhyphen{} Prints the calibration status.

\end{fulllineitems}

\index{Court2Image() (CameraUtils.Camera method)@\spxentry{Court2Image()}\spxextra{CameraUtils.Camera method}}

\begin{fulllineitems}
\phantomsection\label{\detokenize{CameraUtils:id6}}
\pysigstartsignatures
\pysiglinewithargsret{\sphinxbfcode{\sphinxupquote{Court2Image}}}{\sphinxparam{\DUrole{n}{coords}}}{}
\pysigstopsignatures
\sphinxAtStartPar
Convert court coordinates to image coordinates using the homography matrix.

\sphinxAtStartPar
This method maps 2D coordinates from the court plane to 2D image coordinates
using the homography matrix. It is useful for projecting points from the court
space onto the image plane.
\begin{quote}\begin{description}
\sphinxlineitem{Parameters}
\sphinxAtStartPar
\sphinxstyleliteralstrong{\sphinxupquote{coords}} (\sphinxstyleliteralemphasis{\sphinxupquote{numpy.ndarray}}) \textendash{} A 2D numpy array of shape (n, 2) representing coordinates on the court plane in the format {[}x, y{]}.

\sphinxlineitem{Returns}
\sphinxAtStartPar
A 2D numpy array of shape (n, 2) representing the image coordinates in the format {[}u, v{]}.

\sphinxlineitem{Return type}
\sphinxAtStartPar
numpy.ndarray

\end{description}\end{quote}
\subsubsection*{Notes}
\begin{itemize}
\item {} 
\sphinxAtStartPar
The method appends a column of ones to the court coordinates to convert them to homogeneous coordinates.

\item {} 
\sphinxAtStartPar
The homography matrix \sphinxtitleref{self.H\_mtx} must be set before calling this method. If \sphinxtitleref{self.H\_mtx} is \sphinxtitleref{None}, the method prints an error message and returns \sphinxtitleref{None}.

\item {} 
\sphinxAtStartPar
The coordinates are transformed using the homography matrix and normalized to obtain the (u, v) coordinates on the image plane.

\end{itemize}


\begin{sphinxseealso}{See also:}
\begin{description}
\sphinxlineitem{{\hyperref[\detokenize{CameraUtils:id14}]{\sphinxcrossref{\sphinxcode{\sphinxupquote{Image2Court}}}}}}
\sphinxAtStartPar
Method used to convert image coordinates to court coordinates.

\end{description}


\end{sphinxseealso}


\end{fulllineitems}

\index{ExtCalibration() (CameraUtils.Camera method)@\spxentry{ExtCalibration()}\spxextra{CameraUtils.Camera method}}

\begin{fulllineitems}
\phantomsection\label{\detokenize{CameraUtils:id7}}
\pysigstartsignatures
\pysiglinewithargsret{\sphinxbfcode{\sphinxupquote{ExtCalibration}}}{}{}
\pysigstopsignatures
\sphinxAtStartPar
Perform external camera calibration to compute the transformation matrices between the world and camera coordinates.

\sphinxAtStartPar
This method calculates the rotation and translation vectors using the \sphinxtitleref{cv2.solvePnP} function, which solves the Perspective\sphinxhyphen{}n\sphinxhyphen{}Point problem to find the position and orientation of the camera relative to the world coordinate system. It then computes the world\sphinxhyphen{}to\sphinxhyphen{}camera and camera\sphinxhyphen{}to\sphinxhyphen{}world transformation matrices and stores them as instance attributes. Additionally, it computes the homography matrix for the plane to plane mapping of the court plane to the image plane.


\subsection{Returns:}
\label{\detokenize{CameraUtils:returns}}
\sphinxAtStartPar
None


\subsection{Raises:}
\label{\detokenize{CameraUtils:raises}}\begin{description}
\sphinxlineitem{SystemExit}
\sphinxAtStartPar
If the PnP solving process fails, an error message is printed, and the program exits.

\end{description}


\subsection{Notes:}
\label{\detokenize{CameraUtils:notes}}\begin{itemize}
\item {} \begin{description}
\sphinxlineitem{The method sets the following instance attributes:}\begin{itemize}
\item {} \begin{description}
\sphinxlineitem{\sphinxtitleref{self.W2C\_mtx}}{[}numpy.ndarray{]}
\sphinxAtStartPar
4x4 transformation matrix from world coordinates to camera coordinates.

\end{description}

\item {} \begin{description}
\sphinxlineitem{\sphinxtitleref{self.C2W\_mtx}}{[}numpy.ndarray{]}
\sphinxAtStartPar
4x4 transformation matrix from camera coordinates to world coordinates.

\end{description}

\item {} \begin{description}
\sphinxlineitem{\sphinxtitleref{self.EXT\_CAL}}{[}bool{]}
\sphinxAtStartPar
Flag indicating that the external calibration has been successfully completed.

\end{description}

\item {} \begin{description}
\sphinxlineitem{\sphinxtitleref{self.H\_mtx}}{[}numpy.ndarray{]}
\sphinxAtStartPar
Homography matrix for the court plane to the image plane.

\end{description}

\end{itemize}

\end{description}

\item {} \begin{description}
\sphinxlineitem{This method relies on the instance attributes:}\begin{itemize}
\item {} \begin{description}
\sphinxlineitem{\sphinxtitleref{self.corr\_world\_points}}{[}list of tuples{]}
\sphinxAtStartPar
List of corresponding points in the world coordinates.

\end{description}

\item {} \begin{description}
\sphinxlineitem{\sphinxtitleref{self.corr\_image\_points}}{[}list of tuples{]}
\sphinxAtStartPar
List of corresponding points in the image coordinates.

\end{description}

\item {} \begin{description}
\sphinxlineitem{\sphinxtitleref{self.new\_mtx}}{[}numpy.ndarray{]}
\sphinxAtStartPar
Camera matrix after undistorsion.

\end{description}

\item {} \begin{description}
\sphinxlineitem{\sphinxtitleref{self.dist}}{[}numpy.ndarray{]}
\sphinxAtStartPar
Distortion coefficients.

\end{description}

\end{itemize}

\end{description}

\end{itemize}

\end{fulllineitems}

\index{FindHomography() (CameraUtils.Camera method)@\spxentry{FindHomography()}\spxextra{CameraUtils.Camera method}}

\begin{fulllineitems}
\phantomsection\label{\detokenize{CameraUtils:id8}}
\pysigstartsignatures
\pysiglinewithargsret{\sphinxbfcode{\sphinxupquote{FindHomography}}}{}{}
\pysigstopsignatures
\sphinxAtStartPar
Compute the homography matrix from world coordinates (court plane) to image coordinates.

\sphinxAtStartPar
This method calculates the homography matrix using corresponding world and image
points on a plane. It utilizes the RANSAC algorithm to robustly estimate the homography.
\subsubsection*{Notes}
\begin{itemize}
\item {} 
\sphinxAtStartPar
The method requires at least 4 corresponding points in both the world and image
coordinate systems to compute the homography matrix. If fewer than 4 points are
provided, an error message is printed and the method returns \sphinxtitleref{None}.

\item {} 
\sphinxAtStartPar
The computed homography matrix \sphinxtitleref{H} is stored in the \sphinxtitleref{self.H\_mtx} attribute.

\item {} 
\sphinxAtStartPar
The RANSAC algorithm is used with a reprojection threshold of 5 and a confidence
level of 0.9 to handle outliers in the point correspondences.

\end{itemize}
\subsubsection*{Example}

\sphinxAtStartPar
To compute the homography matrix, ensure that \sphinxtitleref{self.corr\_world\_points} and
\sphinxtitleref{self.corr\_image\_points} are properly set with at least 4 corresponding points.
Then call:

\begin{sphinxVerbatim}[commandchars=\\\{\}]
\PYG{n+nb+bp}{self}\PYG{o}{.}\PYG{n}{FindHomography}\PYG{p}{(}\PYG{p}{)}
\end{sphinxVerbatim}

\sphinxAtStartPar
where \sphinxtitleref{self} is an instance of the class containing this method.


\begin{sphinxseealso}{See also:}
\begin{description}
\sphinxlineitem{\sphinxcode{\sphinxupquote{cv2.findHomography}}}
\sphinxAtStartPar
OpenCV function used to compute the homography matrix.

\end{description}


\end{sphinxseealso}


\end{fulllineitems}

\index{GetCornersFromSamples() (CameraUtils.Camera method)@\spxentry{GetCornersFromSamples()}\spxextra{CameraUtils.Camera method}}

\begin{fulllineitems}
\phantomsection\label{\detokenize{CameraUtils:id9}}
\pysigstartsignatures
\pysiglinewithargsret{\sphinxbfcode{\sphinxupquote{GetCornersFromSamples}}}{\sphinxparam{\DUrole{n}{sample\_folder}}}{}
\pysigstopsignatures
\sphinxAtStartPar
Reads existing images from a folder of existing samples to use for camera calibration.

\sphinxAtStartPar
This method reads sample images and detects chessboard corners
in these frames. It then saves the frames and the detected points
to be used for camera calibration.
\begin{quote}\begin{description}
\sphinxlineitem{Parameters}
\sphinxAtStartPar
\sphinxstyleliteralstrong{\sphinxupquote{sample\_folder}} (\sphinxstyleliteralemphasis{\sphinxupquote{str}}) \textendash{} The directory where the sampled frames are retrieved.

\sphinxlineitem{Raises}
\sphinxAtStartPar
\sphinxstyleliteralstrong{\sphinxupquote{FileNotFoundError}} \textendash{} If the specified sample folder does not exist.

\end{description}\end{quote}
\subsubsection*{Notes}
\begin{itemize}
\item {} \begin{description}
\sphinxlineitem{The images are expected to be named in a specific format:}
\sphinxAtStartPar
“out\{camera\_number\}F\{format\}”,
where ‘camera\_number’ is the identifier
of the camera and ‘format’ is defined in the configuration
(example \sphinxtitleref{out3F.jpg} for camera number 3).

\end{description}

\item {} 
\sphinxAtStartPar
The method uses a chessboard pattern to detect corners in the frames.

\item {} \begin{description}
\sphinxlineitem{It divides the frame into quadrants and saves a specified number of}
\sphinxAtStartPar
sample frames from each quadrant:
\sphinxhyphen{} TL: Top\sphinxhyphen{}Left
\sphinxhyphen{} TR: Top\sphinxhyphen{}Right
\sphinxhyphen{} BL: Bottom\sphinxhyphen{}Left
\sphinxhyphen{} BR: Bottom\sphinxhyphen{}Right

\end{description}

\item {} \begin{description}
\sphinxlineitem{The method stops when the required number of samples from all}
\sphinxAtStartPar
quadrants have been collected. The required number of steps can be
set in the setting script NUM\_OF\_SAMPLES.

\end{description}

\end{itemize}

\end{fulllineitems}

\index{GetCorrespondences() (CameraUtils.Camera method)@\spxentry{GetCorrespondences()}\spxextra{CameraUtils.Camera method}}

\begin{fulllineitems}
\phantomsection\label{\detokenize{CameraUtils:id10}}
\pysigstartsignatures
\pysiglinewithargsret{\sphinxbfcode{\sphinxupquote{GetCorrespondences}}}{\sphinxparam{\DUrole{n}{world\_points}}\sphinxparamcomma \sphinxparam{\DUrole{n}{image\_path}}}{}
\pysigstopsignatures
\sphinxAtStartPar
Obtain correspondences between world points and image points through a manual user interface.


\subsection{Parameters:}
\label{\detokenize{CameraUtils:parameters}}\begin{description}
\sphinxlineitem{world\_points}{[}list of tuples{]}
\sphinxAtStartPar
List of world points (x, y, z) that need to be mapped to the image points (They must lie on the court plane, so z=0!).

\sphinxlineitem{image\_path}{[}str{]}
\sphinxAtStartPar
Path to the image file in which correspondences need to be identified.

\end{description}


\subsection{Returns:}
\label{\detokenize{CameraUtils:id11}}\begin{description}
\sphinxlineitem{coords}{[}list of lists{]}
\sphinxAtStartPar
List of image coordinates corresponding to the world points. If acquisition is skipped or terminated, returns an empty list.
Each coordinate is either a list of two floats {[}x, y{]} or {[}NaN, NaN{]} if the point is skipped.

\end{description}


\subsection{Notes:}
\label{\detokenize{CameraUtils:id12}}
\sphinxAtStartPar
This function allows the user to manually select corresponding points in the image using a GUI. The user can interact with the GUI to select points, skip points, or terminate the acquisition process.

\sphinxAtStartPar
If the camera is not calibrated, the function will print a message and return without acquiring any correspondences.

\sphinxAtStartPar
The user can interact with the GUI as follows:
\sphinxhyphen{} Right\sphinxhyphen{}click on the image to select a point.
\sphinxhyphen{} Press the space bar to skip the current point.
\sphinxhyphen{} Press ‘n’ to skip the calibration for the current camera.
\sphinxhyphen{} Press ‘t’ to terminate the point acquisition process.

\sphinxAtStartPar
This method uses OpenCV for image processing and Matplotlib for the GUI.


\subsection{Internal Processing:}
\label{\detokenize{CameraUtils:internal-processing}}\begin{enumerate}
\sphinxsetlistlabels{\arabic}{enumi}{enumii}{}{.}%
\item {} 
\sphinxAtStartPar
Reads the court plan image and the input image.

\item {} 
\sphinxAtStartPar
Undistort the input image based on the camera calibration parameters.

\item {} 
\sphinxAtStartPar
Optionally crops the image based on the region of interest (ROI).

\item {} 
\sphinxAtStartPar
Transforms world points to court plan coordinates.

\item {} 
\sphinxAtStartPar
Initiates a GUI for the user to select corresponding points.

\item {} 
\sphinxAtStartPar
Handles user interactions via mouse clicks and keyboard events.

\item {} 
\sphinxAtStartPar
Returns the list of acquired coordinates or an empty list if acquisition is skipped/terminated.

\end{enumerate}

\end{fulllineitems}

\index{GetSampleFrames() (CameraUtils.Camera method)@\spxentry{GetSampleFrames()}\spxextra{CameraUtils.Camera method}}

\begin{fulllineitems}
\phantomsection\label{\detokenize{CameraUtils:id13}}
\pysigstartsignatures
\pysiglinewithargsret{\sphinxbfcode{\sphinxupquote{GetSampleFrames}}}{\sphinxparam{\DUrole{n}{video\_dir}}\sphinxparamcomma \sphinxparam{\DUrole{n}{frame\_skip}}\sphinxparamcomma \sphinxparam{\DUrole{n}{out\_dir}}}{}
\pysigstopsignatures
\sphinxAtStartPar
Extracts sample frames from a video to use for camera calibration.

\sphinxAtStartPar
This method reads a video file, samples frames at specified intervals,
and detects chessboard corners in these frames. It then saves the frames
and the detected points to be used for camera calibration.
\begin{quote}\begin{description}
\sphinxlineitem{Parameters}\begin{itemize}
\item {} 
\sphinxAtStartPar
\sphinxstyleliteralstrong{\sphinxupquote{video\_dir}} (\sphinxstyleliteralemphasis{\sphinxupquote{str}}) \textendash{} The directory where the video file is located.

\item {} 
\sphinxAtStartPar
\sphinxstyleliteralstrong{\sphinxupquote{frame\_skip}} (\sphinxstyleliteralemphasis{\sphinxupquote{int}}) \textendash{} The number of frames to skip between each sampled frame.

\item {} 
\sphinxAtStartPar
\sphinxstyleliteralstrong{\sphinxupquote{out\_dir}} (\sphinxstyleliteralemphasis{\sphinxupquote{str}}) \textendash{} The directory where the sampled frames will be saved.

\end{itemize}

\sphinxlineitem{Raises}\begin{itemize}
\item {} 
\sphinxAtStartPar
\sphinxstyleliteralstrong{\sphinxupquote{FileNotFoundError}} \textendash{} If the specified video file does not exist.

\item {} 
\sphinxAtStartPar
\sphinxstyleliteralstrong{\sphinxupquote{RuntimeError}} \textendash{} If the video file cannot be opened or read.

\end{itemize}

\end{description}\end{quote}
\subsubsection*{Notes}
\begin{itemize}
\item {} \begin{description}
\sphinxlineitem{The video file is expected to be named in a specific format:}
\sphinxAtStartPar
“out\{camera\_number\}F\{format\}”,
where ‘camera\_number’ is the identifier
of the camera and ‘format’ is defined in the configuration
(example ‘out3F.mp4’ for camera number 3)

\end{description}

\item {} 
\sphinxAtStartPar
The method uses a chessboard pattern to detect corners in the frames.

\item {} \begin{description}
\sphinxlineitem{It divides the frame into quadrants and saves a specified number of}
\sphinxAtStartPar
sample frames from each quadrant:
\sphinxhyphen{} TL: Top\sphinxhyphen{}Left
\sphinxhyphen{} TR: Top\sphinxhyphen{}Right
\sphinxhyphen{} BL: Bottom\sphinxhyphen{}Left
\sphinxhyphen{} BR: Bottom\sphinxhyphen{}Right

\end{description}

\item {} \begin{description}
\sphinxlineitem{The method stops when the required number of samples from all}
\sphinxAtStartPar
quadrants have been collected. The required number of steps can be
set in the setting script NUM\_OF\_SAMPLES.

\end{description}

\end{itemize}

\end{fulllineitems}

\index{Image2Court() (CameraUtils.Camera method)@\spxentry{Image2Court()}\spxextra{CameraUtils.Camera method}}

\begin{fulllineitems}
\phantomsection\label{\detokenize{CameraUtils:id14}}
\pysigstartsignatures
\pysiglinewithargsret{\sphinxbfcode{\sphinxupquote{Image2Court}}}{\sphinxparam{\DUrole{n}{coords}}}{}
\pysigstopsignatures
\sphinxAtStartPar
Convert image coordinates to court coordinates using the homography matrix.

\sphinxAtStartPar
This method transforms 2D image coordinates into coordinates on the court plane
by applying the inverse of the homography matrix. It is useful for mapping points
from the image space to a corresponding position on the court.
\begin{quote}\begin{description}
\sphinxlineitem{Parameters}
\sphinxAtStartPar
\sphinxstyleliteralstrong{\sphinxupquote{coords}} (\sphinxstyleliteralemphasis{\sphinxupquote{numpy.ndarray}}) \textendash{} A 2D numpy array of shape (n, 2) representing image coordinates in the format {[}u, v{]}.

\sphinxlineitem{Returns}
\sphinxAtStartPar
A 2D numpy array of shape (n, 2) representing the coordinates on the court plane in the format {[}x, y{]}.

\sphinxlineitem{Return type}
\sphinxAtStartPar
numpy.ndarray

\end{description}\end{quote}
\subsubsection*{Notes}
\begin{itemize}
\item {} 
\sphinxAtStartPar
The method appends a column of ones to the image coordinates to convert them to homogeneous coordinates.

\item {} 
\sphinxAtStartPar
The homography matrix \sphinxtitleref{self.H\_mtx} must be set before calling this method. If \sphinxtitleref{self.H\_mtx} is \sphinxtitleref{None}, the method prints an error message and returns \sphinxtitleref{None}.

\item {} 
\sphinxAtStartPar
The inverse of the homography matrix is used to transform the coordinates to the court plane.

\item {} 
\sphinxAtStartPar
The resulting coordinates are normalized to obtain the (x, y) coordinates on the court.

\end{itemize}


\begin{sphinxseealso}{See also:}
\begin{description}
\sphinxlineitem{{\hyperref[\detokenize{CameraUtils:id8}]{\sphinxcrossref{\sphinxcode{\sphinxupquote{FindHomography}}}}}}
\sphinxAtStartPar
Method used to compute the homography matrix which is required for this conversion.

\sphinxlineitem{{\hyperref[\detokenize{CameraUtils:id6}]{\sphinxcrossref{\sphinxcode{\sphinxupquote{Court2Image}}}}}}
\sphinxAtStartPar
Inverse method.

\end{description}


\end{sphinxseealso}


\end{fulllineitems}

\index{LoadCamera() (CameraUtils.Camera class method)@\spxentry{LoadCamera()}\spxextra{CameraUtils.Camera class method}}

\begin{fulllineitems}
\phantomsection\label{\detokenize{CameraUtils:CameraUtils.Camera.LoadCamera}}
\pysigstartsignatures
\pysiglinewithargsret{\sphinxbfcode{\sphinxupquote{classmethod\DUrole{w}{ }}}\sphinxbfcode{\sphinxupquote{LoadCamera}}}{\sphinxparam{\DUrole{n}{param\_dir}}\sphinxparamcomma \sphinxparam{\DUrole{n}{camera}}}{}
\pysigstopsignatures
\sphinxAtStartPar
Load camera parameters from a .pkl file.

\sphinxAtStartPar
This method deserializes the Camera object from a file in the specified
directory.
\begin{quote}\begin{description}
\sphinxlineitem{Parameters}\begin{itemize}
\item {} 
\sphinxAtStartPar
\sphinxstyleliteralstrong{\sphinxupquote{param\_dir}} (\sphinxstyleliteralemphasis{\sphinxupquote{str}}) \textendash{} The directory where the parameters are stored.

\item {} 
\sphinxAtStartPar
\sphinxstyleliteralstrong{\sphinxupquote{camera}} (\sphinxstyleliteralemphasis{\sphinxupquote{str}}) \textendash{} The identifier for the camera to be loaded.

\end{itemize}

\sphinxlineitem{Returns}
\sphinxAtStartPar
The Camera object with the loaded parameters.

\sphinxlineitem{Return type}
\sphinxAtStartPar
{\hyperref[\detokenize{CameraUtils:CameraUtils.Camera}]{\sphinxcrossref{Camera}}}

\sphinxlineitem{Raises}
\sphinxAtStartPar
\sphinxstyleliteralstrong{\sphinxupquote{FileNotFoundError}} \textendash{} If the specified parameter directory does not exist.

\end{description}\end{quote}
\subsubsection*{Notes}
\begin{itemize}
\item {} 
\sphinxAtStartPar
The method expects the file to be named “Camera\_\{camera\_number\}.pkl”
where \sphinxtitleref{\{camera\_number\}} is the identifier of the camera.

\end{itemize}

\end{fulllineitems}

\index{NewOptimalCameraMatrix() (CameraUtils.Camera method)@\spxentry{NewOptimalCameraMatrix()}\spxextra{CameraUtils.Camera method}}

\begin{fulllineitems}
\phantomsection\label{\detokenize{CameraUtils:id15}}
\pysigstartsignatures
\pysiglinewithargsret{\sphinxbfcode{\sphinxupquote{NewOptimalCameraMatrix}}}{}{}
\pysigstopsignatures
\sphinxAtStartPar
Refine the camera matrix for the undistortion process.

\sphinxAtStartPar
This method computes a new optimal camera matrix based on the current
camera matrix and distortion coefficients. It adjusts the camera matrix
to improve the undistortion of images.

\sphinxAtStartPar
Depending on whether a fisheye lens is used or a standard lens, the
method applies different algorithms to compute the new camera matrix:
\begin{itemize}
\item {} 
\sphinxAtStartPar
For fisheye lenses, it uses
\sphinxcode{\sphinxupquote{cv2.fisheye.estimateNewCameraMatrixForUndistortRectify()}}.

\item {} 
\sphinxAtStartPar
For standard lenses, it uses \sphinxcode{\sphinxupquote{cv2.getOptimalNewCameraMatrix()}}.

\end{itemize}

\sphinxAtStartPar
The computed camera matrix is stored in \sphinxtitleref{self.new\_mtx}. In the case of
standard lenses, the method also updates \sphinxtitleref{self.roi} with the region of
interest.
\subsubsection*{Notes}
\begin{itemize}
\item {} 
\sphinxAtStartPar
Ensure that the \sphinxtitleref{self.img\_size}, \sphinxtitleref{self.mtx}, and \sphinxtitleref{self.dist} are
correctly set before calling this method.

\item {} 
\sphinxAtStartPar
The \sphinxtitleref{self.FISHEYE} flag determines whether to use the fisheye or
standard lens processing approach.

\end{itemize}

\end{fulllineitems}

\index{PlotCamera() (CameraUtils.Camera method)@\spxentry{PlotCamera()}\spxextra{CameraUtils.Camera method}}

\begin{fulllineitems}
\phantomsection\label{\detokenize{CameraUtils:id16}}
\pysigstartsignatures
\pysiglinewithargsret{\sphinxbfcode{\sphinxupquote{PlotCamera}}}{\sphinxparam{\DUrole{n}{ax}}}{}
\pysigstopsignatures
\sphinxAtStartPar
Plot the camera’s position and direction on a 3D axis.

\sphinxAtStartPar
This method visualizes the camera’s approximate position and estimated position
on a 3D plot. It also displays a line indicating the camera’s direction vector and
a dashed line connecting the approximate and estimated positions.
\begin{quote}\begin{description}
\sphinxlineitem{Parameters}
\sphinxAtStartPar
\sphinxstyleliteralstrong{\sphinxupquote{ax}} (\sphinxstyleliteralemphasis{\sphinxupquote{matplotlib.axes.\_axes.Axes}}) \textendash{} The 3D axis object on which to plot the camera information.

\end{description}\end{quote}
\subsubsection*{Notes}

\sphinxAtStartPar
This method is intended to bella called by the PlotMultipleCameras() class method.


\begin{sphinxseealso}{See also:}

\sphinxAtStartPar
{\hyperref[\detokenize{CameraUtils:id18}]{\sphinxcrossref{\sphinxcode{\sphinxupquote{PlotMultipleCameras}}}}}


\end{sphinxseealso}


\end{fulllineitems}

\index{PlotCamera2D() (CameraUtils.Camera method)@\spxentry{PlotCamera2D()}\spxextra{CameraUtils.Camera method}}

\begin{fulllineitems}
\phantomsection\label{\detokenize{CameraUtils:id17}}
\pysigstartsignatures
\pysiglinewithargsret{\sphinxbfcode{\sphinxupquote{PlotCamera2D}}}{\sphinxparam{\DUrole{n}{ax}}}{}
\pysigstopsignatures
\sphinxAtStartPar
Plot the camera’s position and direction on a 2D axis over a court image.

\sphinxAtStartPar
This method visualizes the camera’s estimated and approximate positions on a 2D plot
with an overlay of a court image. It also shows the direction of the camera with a line
and connects the estimated and approximate positions with a dashed line.
\begin{quote}\begin{description}
\sphinxlineitem{Parameters}
\sphinxAtStartPar
\sphinxstyleliteralstrong{\sphinxupquote{ax}} (\sphinxstyleliteralemphasis{\sphinxupquote{matplotlib.axes.\_axes.Axes}}) \textendash{} The 2D axis object on which to plot the camera information.

\end{description}\end{quote}
\subsubsection*{Notes}
\begin{itemize}
\item {} 
\sphinxAtStartPar
\sphinxstyleemphasis{This method is intended to bella called by the PlotMultipleCameras2D() class method.}

\item {} 
\sphinxAtStartPar
The court image is loaded from the path specified by \sphinxtitleref{conf.COURT\_IMG\_XL}.

\item {} 
\sphinxAtStartPar
The camera’s estimated position is plotted as a circular marker, while the
approximate position is plotted as a gray downward\sphinxhyphen{}pointing triangle marker.

\item {} 
\sphinxAtStartPar
A dashed red line connects the estimated position to the approximate position to
indicate positional error.

\item {} 
\sphinxAtStartPar
The camera’s direction is shown as a line extending from the estimated position
in the direction of the camera’s field of view.

\item {} 
\sphinxAtStartPar
An annotation with the camera number is placed near the approximate position,
with a text path effect for better visibility.

\end{itemize}


\begin{sphinxseealso}{See also:}

\sphinxAtStartPar
{\hyperref[\detokenize{CameraUtils:id19}]{\sphinxcrossref{\sphinxcode{\sphinxupquote{PlotMultipleCameras2D}}}}}


\end{sphinxseealso}


\end{fulllineitems}

\index{PlotMultipleCameras() (CameraUtils.Camera class method)@\spxentry{PlotMultipleCameras()}\spxextra{CameraUtils.Camera class method}}

\begin{fulllineitems}
\phantomsection\label{\detokenize{CameraUtils:id18}}
\pysigstartsignatures
\pysiglinewithargsret{\sphinxbfcode{\sphinxupquote{classmethod\DUrole{w}{ }}}\sphinxbfcode{\sphinxupquote{PlotMultipleCameras}}}{\sphinxparam{\DUrole{n}{camera\_list}}}{}
\pysigstopsignatures
\sphinxAtStartPar
Plot the positions and orientations of multiple cameras on a 3D plot.

\sphinxAtStartPar
This class method visualizes all cameras in the provided list on a 3D axis. It
includes the following features:
\begin{itemize}
\item {} 
\sphinxAtStartPar
Plots each camera’s approximate and estimated positions.

\item {} 
\sphinxAtStartPar
Draws direction vectors for each camera.

\item {} 
\sphinxAtStartPar
Optionally overlays a court image or displays predefined court points.

\end{itemize}
\begin{quote}\begin{description}
\sphinxlineitem{Parameters}
\sphinxAtStartPar
\sphinxstyleliteralstrong{\sphinxupquote{camera\_list}} (\sphinxstyleliteralemphasis{\sphinxupquote{list}}) \textendash{} 
\sphinxAtStartPar
A list of camera objects, where each object should have the following methods
and attributes:
\sphinxhyphen{} \sphinxtitleref{PlotCamera(ax)}: Method to plot a single camera on the 3D axis.
\sphinxhyphen{} \sphinxtitleref{approximate\_position} (list or array\sphinxhyphen{}like): The known approximate position
\begin{quote}

\sphinxAtStartPar
of the camera (x, y, z).
\end{quote}
\begin{itemize}
\item {} 
\sphinxAtStartPar
\sphinxtitleref{C2W\_mtx} (numpy.ndarray): A 4x4 matrix containing the estimated position
in the last column.

\item {} 
\sphinxAtStartPar
\sphinxtitleref{camera\_number} (int or str): An identifier for the camera.

\item {} 
\sphinxAtStartPar
\sphinxtitleref{pos\_err} (float): The positional error of the camera.

\item {} 
\sphinxAtStartPar
\sphinxtitleref{rep\_err} (float): The re\sphinxhyphen{}projection error of the camera.

\end{itemize}


\sphinxlineitem{Returns}
\sphinxAtStartPar
\begin{itemize}
\item {} 
\sphinxAtStartPar
\sphinxstyleemphasis{matplotlib.figure.Figure}

\item {} 
\sphinxAtStartPar
\sphinxstyleemphasis{The figure object containing the 2D plot with camera positions.}

\end{itemize}


\end{description}\end{quote}
\subsubsection*{Notes}
\begin{itemize}
\item {} 
\sphinxAtStartPar
The method creates a 3D plot with camera positions, direction vectors, and
optional court details.

\item {} 
\sphinxAtStartPar
Court image is plotted if \sphinxtitleref{conf.SHOW\_COURT} is \sphinxtitleref{True}, otherwise, predefined
court points are displayed.

\item {} 
\sphinxAtStartPar
The plot’s axes are labeled and ticked to provide clear context for the camera
positions.

\end{itemize}
\subsubsection*{Example}

\sphinxAtStartPar
To plot multiple cameras:

\begin{sphinxVerbatim}[commandchars=\\\{\}]
\PYG{n}{fig} \PYG{o}{=} \PYG{n}{CameraUtils}\PYG{o}{.}\PYG{n}{Camera}\PYG{o}{.}\PYG{n}{PlotMultipleCameras}\PYG{p}{(}\PYG{n}{camera\PYGZus{}list}\PYG{p}{)}
\end{sphinxVerbatim}

\sphinxAtStartPar
where \sphinxtitleref{Camera} is the class containing this method, and \sphinxtitleref{camera\_list} is a list
of camera objects.

\end{fulllineitems}

\index{PlotMultipleCameras2D() (CameraUtils.Camera class method)@\spxentry{PlotMultipleCameras2D()}\spxextra{CameraUtils.Camera class method}}

\begin{fulllineitems}
\phantomsection\label{\detokenize{CameraUtils:id19}}
\pysigstartsignatures
\pysiglinewithargsret{\sphinxbfcode{\sphinxupquote{classmethod\DUrole{w}{ }}}\sphinxbfcode{\sphinxupquote{PlotMultipleCameras2D}}}{\sphinxparam{\DUrole{n}{camera\_list}}}{}
\pysigstopsignatures
\sphinxAtStartPar
Plot the positions of multiple cameras on a 2D court image.

\sphinxAtStartPar
This class method creates a 2D plot showing the positions of all cameras in the
provided list overlaid on a court image. Each camera’s position is plotted, and
their respective direction vectors are displayed. The court image serves as a
reference for the positions.
\begin{quote}\begin{description}
\sphinxlineitem{Parameters}
\sphinxAtStartPar
\sphinxstyleliteralstrong{\sphinxupquote{camera\_list}} (\sphinxstyleliteralemphasis{\sphinxupquote{list}}) \textendash{} A list of camera objects, where each object should have the \sphinxtitleref{PlotCamera2D(ax)} method to plot its position on the 2D axis.

\sphinxlineitem{Returns}
\sphinxAtStartPar
The figure object containing the 2D plot with camera positions.

\sphinxlineitem{Return type}
\sphinxAtStartPar
matplotlib.figure.Figure

\end{description}\end{quote}
\subsubsection*{Notes}
\begin{itemize}
\item {} 
\sphinxAtStartPar
The court image is loaded from the path specified by \sphinxtitleref{conf.COURT\_IMG\_XL}.

\item {} 
\sphinxAtStartPar
The x and y coordinates are adjusted to match the dimensions of the image.

\item {} 
\sphinxAtStartPar
Each camera is plotted using the \sphinxtitleref{PlotCamera2D} method, which should be defined in the same class as this method.

\item {} 
\sphinxAtStartPar
The axis labels are set to “x” and “y” to indicate the coordinate axes.

\item {} 
\sphinxAtStartPar
The image axis is turned off for a cleaner display of camera positions.

\end{itemize}
\subsubsection*{Example}

\sphinxAtStartPar
To plot multiple cameras on a 2D image:

\begin{sphinxVerbatim}[commandchars=\\\{\}]
\PYG{n}{fig} \PYG{o}{=} \PYG{n}{CameraClass}\PYG{o}{.}\PYG{n}{PlotMultipleCameras2D}\PYG{p}{(}\PYG{n}{camera\PYGZus{}list}\PYG{p}{)}
\end{sphinxVerbatim}

\sphinxAtStartPar
where \sphinxtitleref{CameraClass} is the class containing this method, and \sphinxtitleref{camera\_list} is a list of camera objects.


\begin{sphinxseealso}{See also:}
\begin{description}
\sphinxlineitem{{\hyperref[\detokenize{CameraUtils:id17}]{\sphinxcrossref{\sphinxcode{\sphinxupquote{PlotCamera2D}}}}}}
\sphinxAtStartPar
Method used to plot individual camera positions on a 2D axis.

\end{description}


\end{sphinxseealso}


\end{fulllineitems}

\index{PosError() (CameraUtils.Camera method)@\spxentry{PosError()}\spxextra{CameraUtils.Camera method}}

\begin{fulllineitems}
\phantomsection\label{\detokenize{CameraUtils:id20}}
\pysigstartsignatures
\pysiglinewithargsret{\sphinxbfcode{\sphinxupquote{PosError}}}{}{}
\pysigstopsignatures
\sphinxAtStartPar
Calculate and display the positional error between the approximate and estimated positions.

\sphinxAtStartPar
This method computes the Euclidean distance between the approximate\_position\textasciigrave{} and the position derived from the \sphinxtitleref{C2W\_mtx} matrix. The result is printed and stored in the instance variable \sphinxtitleref{self.pos\_err}.
\subsubsection*{Notes}
\begin{itemize}
\item {} 
\sphinxAtStartPar
\sphinxtitleref{self.approximate\_position} should be a list or array\sphinxhyphen{}like structure containing the
approximate position coordinates (x, y, z).

\item {} 
\sphinxAtStartPar
\sphinxtitleref{self.C2W\_mtx} should be a 4x4 matrix where the last column represents the estimated
position in 3D space.

\item {} 
\sphinxAtStartPar
The Euclidean distance is calculated as the norm of the difference between the
approximate and estimated positions.

\item {} 
\sphinxAtStartPar
The result is printed in meters with three decimal places.

\end{itemize}

\end{fulllineitems}

\index{PrintAttributes() (CameraUtils.Camera method)@\spxentry{PrintAttributes()}\spxextra{CameraUtils.Camera method}}

\begin{fulllineitems}
\phantomsection\label{\detokenize{CameraUtils:id21}}
\pysigstartsignatures
\pysiglinewithargsret{\sphinxbfcode{\sphinxupquote{PrintAttributes}}}{\sphinxparam{\DUrole{n}{skip\_attributes}\DUrole{o}{=}\DUrole{default_value}{{[}\textquotesingle{}obj\_points\textquotesingle{}, \textquotesingle{}img\_points\textquotesingle{}{]}}}}{}
\pysigstopsignatures
\sphinxAtStartPar
Prints all the attributes of the CameraInfo instance, except for those in the skip\_attributes list.
\begin{quote}\begin{description}
\sphinxlineitem{Parameters}
\sphinxAtStartPar
\sphinxstyleliteralstrong{\sphinxupquote{skip\_attributes}} (\sphinxstyleliteralemphasis{\sphinxupquote{list}}\sphinxstyleliteralemphasis{\sphinxupquote{ of }}\sphinxstyleliteralemphasis{\sphinxupquote{str}}\sphinxstyleliteralemphasis{\sphinxupquote{, }}\sphinxstyleliteralemphasis{\sphinxupquote{optional}}) \textendash{} 
\sphinxAtStartPar
A list of attribute names to exclude from printing. Default is:
{[}“obj\_points”,”img\_points”{]}

\sphinxAtStartPar
Input skip\_attributes = None to print all attributes


\end{description}\end{quote}

\end{fulllineitems}

\index{RepError() (CameraUtils.Camera method)@\spxentry{RepError()}\spxextra{CameraUtils.Camera method}}

\begin{fulllineitems}
\phantomsection\label{\detokenize{CameraUtils:CameraUtils.Camera.RepError}}
\pysigstartsignatures
\pysiglinewithargsret{\sphinxbfcode{\sphinxupquote{RepError}}}{}{}
\pysigstopsignatures
\sphinxAtStartPar
Calculate the re\sphinxhyphen{}projection error given the parameters found in calibration.
\subsubsection*{Notes}

\sphinxAtStartPar
The new refined camera matrix is used.

\sphinxAtStartPar
If the camera is not calibrated (\sphinxtitleref{INT\_CAL} is False),
the method will print a message indicating that the re\sphinxhyphen{}projection error
cannot be calculated and return without performing any calculations.
\begin{quote}\begin{description}
\sphinxlineitem{Return type}
\sphinxAtStartPar
None

\end{description}\end{quote}

\end{fulllineitems}

\index{SaveErrors() (CameraUtils.Camera class method)@\spxentry{SaveErrors()}\spxextra{CameraUtils.Camera class method}}

\begin{fulllineitems}
\phantomsection\label{\detokenize{CameraUtils:id22}}
\pysigstartsignatures
\pysiglinewithargsret{\sphinxbfcode{\sphinxupquote{classmethod\DUrole{w}{ }}}\sphinxbfcode{\sphinxupquote{SaveErrors}}}{\sphinxparam{\DUrole{n}{camera\_list}}\sphinxparamcomma \sphinxparam{\DUrole{n}{path}}}{}
\pysigstopsignatures
\sphinxAtStartPar
Save the positional and re\sphinxhyphen{}projection errors of a list of cameras to a CSV file.

\sphinxAtStartPar
This class method generates a CSV file containing error data for each
camera in the provided list. The file is saved at the specified \sphinxtitleref{path}
and includes the following columns:
\begin{itemize}
\item {} 
\sphinxAtStartPar
\sphinxtitleref{CAM}: Camera identifier or number.

\item {} 
\sphinxAtStartPar
\sphinxtitleref{x, y, z}: Approximate position coordinates of the camera.

\item {} 
\sphinxAtStartPar
\sphinxtitleref{est. x, est. y, est. z}: Estimated position coordinates from the camera’s
\sphinxtitleref{C2W\_mtx} matrix.

\item {} 
\sphinxAtStartPar
\sphinxtitleref{distance}: Euclidean distance between the approximate and estimated positions.

\item {} 
\sphinxAtStartPar
\sphinxtitleref{re\sphinxhyphen{}projection error}: Error value representing the difference between the actual
and projected positions, resulting from the intrinsic calibration procedure.

\end{itemize}

\sphinxAtStartPar
The CSV file is formatted with headers and includes error values formatted to two
decimal places for positions and distances, and three decimal places for re\sphinxhyphen{}projection
errors.
\begin{quote}\begin{description}
\sphinxlineitem{Parameters}\begin{itemize}
\item {} 
\sphinxAtStartPar
\sphinxstyleliteralstrong{\sphinxupquote{camera\_list}} (\sphinxstyleliteralemphasis{\sphinxupquote{list}}) \textendash{} 
\sphinxAtStartPar
A list of camera objects, where each object should have the following attributes:
\sphinxhyphen{} \sphinxtitleref{approximate\_position} (list or array\sphinxhyphen{}like): The known approximate position of
\begin{quote}

\sphinxAtStartPar
the camera (x, y, z).
\end{quote}
\begin{itemize}
\item {} 
\sphinxAtStartPar
\sphinxtitleref{C2W\_mtx} (numpy.ndarray): A 4x4 matrix containing the estimated position in
the last column.

\item {} 
\sphinxAtStartPar
\sphinxtitleref{camera\_number} (int or str): An identifier for the camera.

\item {} 
\sphinxAtStartPar
\sphinxtitleref{pos\_err} (float): The positional error of the camera.

\item {} 
\sphinxAtStartPar
\sphinxtitleref{rep\_err} (float): The re\sphinxhyphen{}projection error of the camera.

\end{itemize}


\item {} 
\sphinxAtStartPar
\sphinxstyleliteralstrong{\sphinxupquote{path}} (\sphinxstyleliteralemphasis{\sphinxupquote{str}}) \textendash{} The directory path where the CSV file should be saved.

\end{itemize}

\end{description}\end{quote}
\subsubsection*{Notes}
\begin{itemize}
\item {} 
\sphinxAtStartPar
The method assumes that each camera object in \sphinxtitleref{camera\_list} has the required attributes.

\item {} 
\sphinxAtStartPar
The CSV file will be created with the name “Errors.csv” in the specified directory.

\item {} 
\sphinxAtStartPar
Ensure that the \sphinxtitleref{path} provided is a valid directory path.

\end{itemize}

\end{fulllineitems}

\index{SaveParameters() (CameraUtils.Camera method)@\spxentry{SaveParameters()}\spxextra{CameraUtils.Camera method}}

\begin{fulllineitems}
\phantomsection\label{\detokenize{CameraUtils:id23}}
\pysigstartsignatures
\pysiglinewithargsret{\sphinxbfcode{\sphinxupquote{SaveParameters}}}{\sphinxparam{\DUrole{n}{save\_dir}}}{}
\pysigstopsignatures
\sphinxAtStartPar
Save camera parameters to a .pkl file.

\sphinxAtStartPar
This method serializes the Camera object and saves it to a file in the
specified directory.
\begin{quote}\begin{description}
\sphinxlineitem{Parameters}
\sphinxAtStartPar
\sphinxstyleliteralstrong{\sphinxupquote{save\_dir}} (\sphinxstyleliteralemphasis{\sphinxupquote{str}}) \textendash{} The directory where the parameters will be saved.

\end{description}\end{quote}
\subsubsection*{Notes}
\begin{itemize}
\item {} 
\sphinxAtStartPar
If the folder does not exist, it is created.

\item {} 
\sphinxAtStartPar
The saved file is named “Camera\_\{camera\_number\}.pkl” where

\end{itemize}

\sphinxAtStartPar
\sphinxtitleref{\{camera\_number\}} is the identifier of the camera.

\end{fulllineitems}

\index{TestUndistorsion() (CameraUtils.Camera method)@\spxentry{TestUndistorsion()}\spxextra{CameraUtils.Camera method}}

\begin{fulllineitems}
\phantomsection\label{\detokenize{CameraUtils:id24}}
\pysigstartsignatures
\pysiglinewithargsret{\sphinxbfcode{\sphinxupquote{TestUndistorsion}}}{}{}
\pysigstopsignatures
\sphinxAtStartPar
Test the undistorsion of the camera images.

\sphinxAtStartPar
This method reads a sample image, applies the undistorsion process based on
the calibration parameters, and visualizes the results. It also saves the
undistorted image to a specified directory.
\subsubsection*{Notes}
\begin{itemize}
\item {} 
\sphinxAtStartPar
The sample image is read from a predefined folder.

\item {} 
\sphinxAtStartPar
The method checks if the camera is calibrated before proceeding.

\item {} 
\sphinxAtStartPar
If the camera uses a fisheye lens, \sphinxtitleref{cv2.fisheye.undistortImage} is used for undistorsion.

\item {} 
\sphinxAtStartPar
For standard lenses, \sphinxtitleref{cv2.undistort} is used.

\item {} 
\sphinxAtStartPar
If cropping is enabled, the undistorted image is cropped to the region of interest (ROI).

\item {} 
\sphinxAtStartPar
The original and undistorted images are displayed side by side for comparison.

\end{itemize}
\begin{quote}\begin{description}
\sphinxlineitem{Raises}\begin{itemize}
\item {} 
\sphinxAtStartPar
\sphinxstyleliteralstrong{\sphinxupquote{RuntimeError}} \textendash{} If the camera is not calibrated.

\item {} 
\sphinxAtStartPar
\sphinxstyleliteralstrong{\sphinxupquote{FileNotFoundError}} \textendash{} If the sample image is not found.

\item {} 
\sphinxAtStartPar
\sphinxstyleliteralstrong{\sphinxupquote{Saves}} \textendash{} 

\item {} 
\sphinxAtStartPar
\sphinxstyleliteralstrong{\sphinxupquote{\sphinxhyphen{}\sphinxhyphen{}\sphinxhyphen{}\sphinxhyphen{}\sphinxhyphen{}}} \textendash{} 

\item {} 
\sphinxAtStartPar
\sphinxstyleliteralstrong{\sphinxupquote{\sphinxhyphen{} The undistorted image is saved to a predefined directory with the filename "Cam\{camera\_number\}.jpg".}} \textendash{} 

\end{itemize}

\end{description}\end{quote}

\end{fulllineitems}


\end{fulllineitems}

\index{PlotCameras (class in CameraUtils)@\spxentry{PlotCameras}\spxextra{class in CameraUtils}}

\begin{fulllineitems}
\phantomsection\label{\detokenize{CameraUtils:CameraUtils.PlotCameras}}
\pysigstartsignatures
\pysigline{\sphinxbfcode{\sphinxupquote{class\DUrole{w}{ }}}\sphinxcode{\sphinxupquote{CameraUtils.}}\sphinxbfcode{\sphinxupquote{PlotCameras}}}
\pysigstopsignatures
\sphinxAtStartPar
Handles the plotting of camera views around the court and manages the visualization
of points on the court surface across all camera views.

\sphinxAtStartPar
This class is designed to initialize camera views and axes, add new camera views,
and set up a plot layout with the court plan in the center and camera views around it.
\index{views (CameraUtils.PlotCameras attribute)@\spxentry{views}\spxextra{CameraUtils.PlotCameras attribute}}

\begin{fulllineitems}
\phantomsection\label{\detokenize{CameraUtils:CameraUtils.PlotCameras.views}}
\pysigstartsignatures
\pysigline{\sphinxbfcode{\sphinxupquote{views}}}
\pysigstopsignatures
\sphinxAtStartPar
A dictionary mapping camera identifiers (e.g., “CAM1”, “CAM2”) to instances of
the \sphinxtitleref{Camera} class. This holds all the camera views.
\begin{quote}\begin{description}
\sphinxlineitem{Type}
\sphinxAtStartPar
dict

\end{description}\end{quote}

\end{fulllineitems}

\index{axes (CameraUtils.PlotCameras attribute)@\spxentry{axes}\spxextra{CameraUtils.PlotCameras attribute}}

\begin{fulllineitems}
\phantomsection\label{\detokenize{CameraUtils:CameraUtils.PlotCameras.axes}}
\pysigstartsignatures
\pysigline{\sphinxbfcode{\sphinxupquote{axes}}}
\pysigstopsignatures
\sphinxAtStartPar
A dictionary mapping camera identifiers and “court” to matplotlib axes objects.
These axes are used for plotting the camera views and the court plan.
\begin{quote}\begin{description}
\sphinxlineitem{Type}
\sphinxAtStartPar
dict

\end{description}\end{quote}

\end{fulllineitems}

\index{points (CameraUtils.PlotCameras attribute)@\spxentry{points}\spxextra{CameraUtils.PlotCameras attribute}}

\begin{fulllineitems}
\phantomsection\label{\detokenize{CameraUtils:CameraUtils.PlotCameras.points}}
\pysigstartsignatures
\pysigline{\sphinxbfcode{\sphinxupquote{points}}}
\pysigstopsignatures
\sphinxAtStartPar
An array of shape (n, 2) to store the coordinates of points picked on the court.
\begin{quote}\begin{description}
\sphinxlineitem{Type}
\sphinxAtStartPar
numpy.ndarray

\end{description}\end{quote}

\end{fulllineitems}

\index{court\_plan (CameraUtils.PlotCameras attribute)@\spxentry{court\_plan}\spxextra{CameraUtils.PlotCameras attribute}}

\begin{fulllineitems}
\phantomsection\label{\detokenize{CameraUtils:CameraUtils.PlotCameras.court_plan}}
\pysigstartsignatures
\pysigline{\sphinxbfcode{\sphinxupquote{court\_plan}}}
\pysigstopsignatures
\sphinxAtStartPar
An image representing the court plan, used as a background for plotting.
\begin{quote}\begin{description}
\sphinxlineitem{Type}
\sphinxAtStartPar
numpy.ndarray

\end{description}\end{quote}

\end{fulllineitems}

\index{color\_counter (CameraUtils.PlotCameras attribute)@\spxentry{color\_counter}\spxextra{CameraUtils.PlotCameras attribute}}

\begin{fulllineitems}
\phantomsection\label{\detokenize{CameraUtils:CameraUtils.PlotCameras.color_counter}}
\pysigstartsignatures
\pysigline{\sphinxbfcode{\sphinxupquote{color\_counter}}}
\pysigstopsignatures
\sphinxAtStartPar
A counter used to assign different colors to highlighted points.
\begin{quote}\begin{description}
\sphinxlineitem{Type}
\sphinxAtStartPar
int

\end{description}\end{quote}

\end{fulllineitems}

\index{\_\_init\_\_() (CameraUtils.PlotCameras method)@\spxentry{\_\_init\_\_()}\spxextra{CameraUtils.PlotCameras method}}

\begin{fulllineitems}
\phantomsection\label{\detokenize{CameraUtils:CameraUtils.PlotCameras.__init__}}
\pysigstartsignatures
\pysiglinewithargsret{\sphinxbfcode{\sphinxupquote{\_\_init\_\_}}}{}{}
\pysigstopsignatures
\sphinxAtStartPar
Initializes the \sphinxtitleref{PlotCameras} object with empty camera views and axes, and loads
the court plan image.

\end{fulllineitems}

\index{AddView() (CameraUtils.PlotCameras method)@\spxentry{AddView()}\spxextra{CameraUtils.PlotCameras method}}

\begin{fulllineitems}
\phantomsection\label{\detokenize{CameraUtils:CameraUtils.PlotCameras.AddView}}
\pysigstartsignatures
\pysiglinewithargsret{\sphinxbfcode{\sphinxupquote{AddView}}}{\sphinxparam{\DUrole{n}{camera}}}{}
\pysigstopsignatures
\sphinxAtStartPar
Adds a new camera view to the \sphinxtitleref{views} dictionary.

\end{fulllineitems}

\index{InitPlot() (CameraUtils.PlotCameras method)@\spxentry{InitPlot()}\spxextra{CameraUtils.PlotCameras method}}

\begin{fulllineitems}
\phantomsection\label{\detokenize{CameraUtils:CameraUtils.PlotCameras.InitPlot}}
\pysigstartsignatures
\pysiglinewithargsret{\sphinxbfcode{\sphinxupquote{InitPlot}}}{}{}
\pysigstopsignatures
\sphinxAtStartPar
Initializes the plot layout with the court plan in the middle and camera views
around it. Configures the layout and sets up the plotting environment.

\end{fulllineitems}

\index{PlotImages() (CameraUtils.PlotCameras method)@\spxentry{PlotImages()}\spxextra{CameraUtils.PlotCameras method}}

\begin{fulllineitems}
\phantomsection\label{\detokenize{CameraUtils:CameraUtils.PlotCameras.PlotImages}}
\pysigstartsignatures
\pysiglinewithargsret{\sphinxbfcode{\sphinxupquote{PlotImages}}}{}{}
\pysigstopsignatures
\sphinxAtStartPar
Plots the images for all camera views on their respective axes.

\end{fulllineitems}

\index{ShowViews() (CameraUtils.PlotCameras method)@\spxentry{ShowViews()}\spxextra{CameraUtils.PlotCameras method}}

\begin{fulllineitems}
\phantomsection\label{\detokenize{CameraUtils:CameraUtils.PlotCameras.ShowViews}}
\pysigstartsignatures
\pysiglinewithargsret{\sphinxbfcode{\sphinxupquote{ShowViews}}}{}{}
\pysigstopsignatures
\sphinxAtStartPar
Displays the plotted views and connects mouse and keyboard event handlers.

\end{fulllineitems}

\index{\_on\_click() (CameraUtils.PlotCameras method)@\spxentry{\_on\_click()}\spxextra{CameraUtils.PlotCameras method}}

\begin{fulllineitems}
\phantomsection\label{\detokenize{CameraUtils:CameraUtils.PlotCameras._on_click}}
\pysigstartsignatures
\pysiglinewithargsret{\sphinxbfcode{\sphinxupquote{\_on\_click}}}{\sphinxparam{\DUrole{n}{event}}}{}
\pysigstopsignatures
\sphinxAtStartPar
Handles mouse click events to record points on the court and save them to the points attribute.

\end{fulllineitems}



\begin{fulllineitems}

\pysigstartsignatures
\pysigline{\sphinxbfcode{\sphinxupquote{\_on\_key(event):}}}
\pysigstopsignatures
\sphinxAtStartPar
Handles keyboard events to clear points when ‘c’ is pressed.

\end{fulllineitems}



\begin{fulllineitems}

\pysigstartsignatures
\pysigline{\sphinxbfcode{\sphinxupquote{DrawPoints():}}}
\pysigstopsignatures
\sphinxAtStartPar
Draws the most recently added point on the images and the court plan.

\end{fulllineitems}



\begin{fulllineitems}

\pysigstartsignatures
\pysigline{\sphinxbfcode{\sphinxupquote{ClearPoints():}}}
\pysigstopsignatures
\sphinxAtStartPar
Clears all recorded points from the \sphinxtitleref{points} attribute and removes them from the plot.

\end{fulllineitems}

\index{AddView() (CameraUtils.PlotCameras method)@\spxentry{AddView()}\spxextra{CameraUtils.PlotCameras method}}

\begin{fulllineitems}
\phantomsection\label{\detokenize{CameraUtils:id25}}
\pysigstartsignatures
\pysiglinewithargsret{\sphinxbfcode{\sphinxupquote{AddView}}}{\sphinxparam{\DUrole{n}{camera}}}{}
\pysigstopsignatures
\sphinxAtStartPar
Adds a camera view to the \sphinxtitleref{views} dictionary.
\begin{quote}\begin{description}
\sphinxlineitem{Parameters}
\sphinxAtStartPar
\sphinxstyleliteralstrong{\sphinxupquote{camera}} ({\hyperref[\detokenize{CameraUtils:CameraUtils.Camera}]{\sphinxcrossref{\sphinxstyleliteralemphasis{\sphinxupquote{Camera}}}}}) \textendash{} An instance of the \sphinxtitleref{Camera} class that will be added to the \sphinxtitleref{views} dictionary.

\end{description}\end{quote}
\subsubsection*{Notes}

\sphinxAtStartPar
The \sphinxtitleref{camera} parameter should be an instance of the \sphinxtitleref{Camera} class with a
\sphinxtitleref{camera\_number} attribute that corresponds to the camera identifier (e.g., “CAM1”).

\end{fulllineitems}

\index{ClearPoints() (CameraUtils.PlotCameras method)@\spxentry{ClearPoints()}\spxextra{CameraUtils.PlotCameras method}}

\begin{fulllineitems}
\phantomsection\label{\detokenize{CameraUtils:CameraUtils.PlotCameras.ClearPoints}}
\pysigstartsignatures
\pysiglinewithargsret{\sphinxbfcode{\sphinxupquote{ClearPoints}}}{}{}
\pysigstopsignatures
\sphinxAtStartPar
Clears all recorded points from the \sphinxtitleref{points} attribute and removes them from the plot.

\sphinxAtStartPar
This method resets the \sphinxtitleref{points} attribute to an empty array and iterates through all axes to remove plotted points. The figure is then updated to reflect the cleared points.

\end{fulllineitems}

\index{DrawPoints() (CameraUtils.PlotCameras method)@\spxentry{DrawPoints()}\spxextra{CameraUtils.PlotCameras method}}

\begin{fulllineitems}
\phantomsection\label{\detokenize{CameraUtils:CameraUtils.PlotCameras.DrawPoints}}
\pysigstartsignatures
\pysiglinewithargsret{\sphinxbfcode{\sphinxupquote{DrawPoints}}}{}{}
\pysigstopsignatures
\sphinxAtStartPar
Draws the most recently added point on the images and the court plan.

\sphinxAtStartPar
This method plots the last added point on all camera views and the court plan.
It projects the point from court coordinates to image coordinates for camera views, and to the court plan for visualization. Points outside the view or court plan are excluded. The color of the points is determined by \sphinxtitleref{color\_counter}.
\subsubsection*{Notes}

\sphinxAtStartPar
The method updates the plot for each camera view and the court plan, and then increments the \sphinxtitleref{color\_counter} for distinguishing multiple points.

\end{fulllineitems}

\index{InitPlot() (CameraUtils.PlotCameras method)@\spxentry{InitPlot()}\spxextra{CameraUtils.PlotCameras method}}

\begin{fulllineitems}
\phantomsection\label{\detokenize{CameraUtils:id26}}
\pysigstartsignatures
\pysiglinewithargsret{\sphinxbfcode{\sphinxupquote{InitPlot}}}{}{}
\pysigstopsignatures
\sphinxAtStartPar
Initializes the plot layout with a court plan in the middle and camera views around it.

\sphinxAtStartPar
This method sets up a matplotlib figure with a grid layout, placing the court plan image in the center and arranging the camera views around it. It also configures the axes for the court plan and each camera view, and sets a figure\sphinxhyphen{}wide title with instructions for interacting with the plot.
\subsubsection*{Notes}

\sphinxAtStartPar
The layout consists of a 4x3 grid, with camera views placed in locations reflecting their real distribution around the court, which is displayed in the central subplot.

\end{fulllineitems}

\index{PlotImages() (CameraUtils.PlotCameras method)@\spxentry{PlotImages()}\spxextra{CameraUtils.PlotCameras method}}

\begin{fulllineitems}
\phantomsection\label{\detokenize{CameraUtils:id27}}
\pysigstartsignatures
\pysiglinewithargsret{\sphinxbfcode{\sphinxupquote{PlotImages}}}{}{}
\pysigstopsignatures
\sphinxAtStartPar
Plots the images for all camera views on their respective axes.

\sphinxAtStartPar
This method reads images from the specified paths, undistorts them based on calibration parameters, resizes the images to improve plot rendering performance, and then displays them on their respective axes in the matplotlib figure. Each camera view is annotated with its identifier.

\sphinxAtStartPar
The images are rescaled by a factor defined in \sphinxtitleref{conf.SCALE} to enhance plotting performance. Camera identifiers are added as text annotations on the images.

\end{fulllineitems}

\index{ShowViews() (CameraUtils.PlotCameras method)@\spxentry{ShowViews()}\spxextra{CameraUtils.PlotCameras method}}

\begin{fulllineitems}
\phantomsection\label{\detokenize{CameraUtils:id28}}
\pysigstartsignatures
\pysiglinewithargsret{\sphinxbfcode{\sphinxupquote{ShowViews}}}{}{}
\pysigstopsignatures
\sphinxAtStartPar
Displays the plotted views and connects mouse and keyboard event handlers.

\sphinxAtStartPar
This method activates the interactive GUI for highlighting points across camera views. It sets up event handlers for mouse clicks and keyboard presses, allowing users to interact with the plot, highlight points, and clear them using specific key commands. The GUI is displayed using \sphinxtitleref{plt.show()}.

\end{fulllineitems}


\end{fulllineitems}



\section{Functions}
\label{\detokenize{CameraUtils:functions}}\index{GetFrame() (in module CameraUtils)@\spxentry{GetFrame()}\spxextra{in module CameraUtils}}

\begin{fulllineitems}
\phantomsection\label{\detokenize{CameraUtils:CameraUtils.GetFrame}}
\pysigstartsignatures
\pysiglinewithargsret{\sphinxcode{\sphinxupquote{CameraUtils.}}\sphinxbfcode{\sphinxupquote{GetFrame}}}{\sphinxparam{\DUrole{n}{video\_folder}}\sphinxparamcomma \sphinxparam{\DUrole{n}{cam\_number}}\sphinxparamcomma \sphinxparam{\DUrole{n}{n}}}{}
\pysigstopsignatures
\sphinxAtStartPar
Extracts and saves a specific frame from a video file.

\sphinxAtStartPar
This function reads the specified frame from a video file associated with a given camera number and saves it as an image file. If the video file cannot be opened or the frame cannot be read, an error message is printed.
\begin{quote}\begin{description}
\sphinxlineitem{Parameters}\begin{itemize}
\item {} 
\sphinxAtStartPar
\sphinxstyleliteralstrong{\sphinxupquote{video\_folder}} (\sphinxstyleliteralemphasis{\sphinxupquote{str}}) \textendash{} The folder where the video files are located.

\item {} 
\sphinxAtStartPar
\sphinxstyleliteralstrong{\sphinxupquote{cam\_number}} (\sphinxstyleliteralemphasis{\sphinxupquote{int}}) \textendash{} The camera number to identify the video file.

\item {} 
\sphinxAtStartPar
\sphinxstyleliteralstrong{\sphinxupquote{n}} (\sphinxstyleliteralemphasis{\sphinxupquote{int}}) \textendash{} The frame number to extract from the video.

\end{itemize}

\sphinxlineitem{Returns}
\sphinxAtStartPar
None

\sphinxlineitem{Return type}
\sphinxAtStartPar
None

\sphinxlineitem{Raises}
\sphinxAtStartPar
\sphinxstyleliteralstrong{\sphinxupquote{ValueError}} \textendash{} If the video file cannot be opened or the frame cannot be read.

\end{description}\end{quote}

\end{fulllineitems}

\index{PrintMtx() (in module CameraUtils)@\spxentry{PrintMtx()}\spxextra{in module CameraUtils}}

\begin{fulllineitems}
\phantomsection\label{\detokenize{CameraUtils:CameraUtils.PrintMtx}}
\pysigstartsignatures
\pysiglinewithargsret{\sphinxcode{\sphinxupquote{CameraUtils.}}\sphinxbfcode{\sphinxupquote{PrintMtx}}}{\sphinxparam{\DUrole{n}{mtx}}}{}
\pysigstopsignatures
\sphinxAtStartPar
Print a matrix in human readable format

\end{fulllineitems}

\index{RW2courtIMG() (in module CameraUtils)@\spxentry{RW2courtIMG()}\spxextra{in module CameraUtils}}

\begin{fulllineitems}
\phantomsection\label{\detokenize{CameraUtils:CameraUtils.RW2courtIMG}}
\pysigstartsignatures
\pysiglinewithargsret{\sphinxcode{\sphinxupquote{CameraUtils.}}\sphinxbfcode{\sphinxupquote{RW2courtIMG}}}{\sphinxparam{\DUrole{n}{RW\_point}}\sphinxparamcomma \sphinxparam{\DUrole{n}{scale}}\sphinxparamcomma \sphinxparam{\DUrole{n}{RF\_Img}}}{}
\pysigstopsignatures
\sphinxAtStartPar
Transforms a point from the real\sphinxhyphen{}world court coordinate system to the image coordinate system
for a court plan.
\begin{quote}\begin{description}
\sphinxlineitem{Parameters}\begin{itemize}
\item {} 
\sphinxAtStartPar
\sphinxstyleliteralstrong{\sphinxupquote{RW\_point}} (\sphinxstyleliteralemphasis{\sphinxupquote{np.ndarray}}) \textendash{} A 3D point in the real\sphinxhyphen{}world court reference frame, represented as a numpy array
in homogeneous coordinates, e.g., np.array({[}x, y, 1{]}).

\item {} 
\sphinxAtStartPar
\sphinxstyleliteralstrong{\sphinxupquote{scale}} (\sphinxstyleliteralemphasis{\sphinxupquote{float}}) \textendash{} The scale factor representing the ratio of pixels to meters.
For example, a scale of 50 means 50 pixels correspond to 1 meter.

\item {} 
\sphinxAtStartPar
\sphinxstyleliteralstrong{\sphinxupquote{RF\_Img}} (\sphinxstyleliteralemphasis{\sphinxupquote{list}}\sphinxstyleliteralemphasis{\sphinxupquote{ or }}\sphinxstyleliteralemphasis{\sphinxupquote{tuple}}) \textendash{} A list or tuple containing the (u, v) coordinates of the reference frame origin in the image,
corresponding to the real\sphinxhyphen{}world origin (x, y) of the court.

\end{itemize}

\sphinxlineitem{Returns}
\sphinxAtStartPar
A 2D numpy array with the transformed (u, v) coordinates in the image plane.

\sphinxlineitem{Return type}
\sphinxAtStartPar
np.ndarray

\end{description}\end{quote}
\subsubsection*{Notes}

\sphinxAtStartPar
The function uses a transformation matrix to convert the real\sphinxhyphen{}world court coordinates
to the image court coordinates, applying scaling and translation to the point.

\end{fulllineitems}

\index{SampleFile() (in module CameraUtils)@\spxentry{SampleFile()}\spxextra{in module CameraUtils}}

\begin{fulllineitems}
\phantomsection\label{\detokenize{CameraUtils:CameraUtils.SampleFile}}
\pysigstartsignatures
\pysiglinewithargsret{\sphinxcode{\sphinxupquote{CameraUtils.}}\sphinxbfcode{\sphinxupquote{SampleFile}}}{\sphinxparam{\DUrole{n}{folder}}}{}
\pysigstopsignatures
\sphinxAtStartPar
Randomly selects a file from a specified folder.

\sphinxAtStartPar
This function scans the provided folder for files and randomly selects one from the list of files found. If no files are found, a warning message is printed.
\begin{quote}\begin{description}
\sphinxlineitem{Parameters}
\sphinxAtStartPar
\sphinxstyleliteralstrong{\sphinxupquote{folder}} (\sphinxstyleliteralemphasis{\sphinxupquote{str}}) \textendash{} The path to the folder to scan for files.

\sphinxlineitem{Returns}
\sphinxAtStartPar
The path of the randomly selected file.

\sphinxlineitem{Return type}
\sphinxAtStartPar
str

\sphinxlineitem{Raises}
\sphinxAtStartPar
\sphinxstyleliteralstrong{\sphinxupquote{FileNotFoundError}} \textendash{} If the folder does not contain any files.

\end{description}\end{quote}

\end{fulllineitems}

\index{courtIMG2RW() (in module CameraUtils)@\spxentry{courtIMG2RW()}\spxextra{in module CameraUtils}}

\begin{fulllineitems}
\phantomsection\label{\detokenize{CameraUtils:CameraUtils.courtIMG2RW}}
\pysigstartsignatures
\pysiglinewithargsret{\sphinxcode{\sphinxupquote{CameraUtils.}}\sphinxbfcode{\sphinxupquote{courtIMG2RW}}}{\sphinxparam{\DUrole{n}{img\_pnt}}\sphinxparamcomma \sphinxparam{\DUrole{n}{scale}}\sphinxparamcomma \sphinxparam{\DUrole{n}{RF\_Img}}}{}
\pysigstopsignatures
\sphinxAtStartPar
Transforms a point from the image coordinate system to the real\sphinxhyphen{}world court coordinate system
using a homogenous transformation.
\begin{quote}\begin{description}
\sphinxlineitem{Parameters}\begin{itemize}
\item {} 
\sphinxAtStartPar
\sphinxstyleliteralstrong{\sphinxupquote{img\_pnt}} (\sphinxstyleliteralemphasis{\sphinxupquote{np.ndarray}}) \textendash{} A point in the image coordinate system, represented as a numpy array in homogeneous coordinates,
e.g., np.array({[}{[}u, v, 1{]}{]}).

\item {} 
\sphinxAtStartPar
\sphinxstyleliteralstrong{\sphinxupquote{scale}} (\sphinxstyleliteralemphasis{\sphinxupquote{float}}) \textendash{} The scale factor representing the ratio of pixels to meters.
For example, a scale of 50 means 50 pixels correspond to 1 meter.

\item {} 
\sphinxAtStartPar
\sphinxstyleliteralstrong{\sphinxupquote{RF\_Img}} (\sphinxstyleliteralemphasis{\sphinxupquote{list}}\sphinxstyleliteralemphasis{\sphinxupquote{ or }}\sphinxstyleliteralemphasis{\sphinxupquote{tuple}}) \textendash{} A list or tuple containing the (u, v) coordinates of the image’s reference frame origin,
corresponding to the real\sphinxhyphen{}world origin (x, y) of the court.

\end{itemize}

\sphinxlineitem{Returns}
\sphinxAtStartPar
A 2D numpy array representing the transformed (x, y) coordinates in the real\sphinxhyphen{}world
court coordinate system, in meters.

\sphinxlineitem{Return type}
\sphinxAtStartPar
np.ndarray

\end{description}\end{quote}
\subsubsection*{Notes}

\sphinxAtStartPar
The function applies an inverse transformation, converting image coordinates into
real\sphinxhyphen{}world court coordinates by applying translation and scaling.

\end{fulllineitems}


\sphinxstepscope


\chapter{Resources}
\label{\detokenize{resources:resources}}\label{\detokenize{resources::doc}}
\sphinxAtStartPar
Useful resources used during development of the project.


\section{Links}
\label{\detokenize{resources:links}}\begin{description}
\sphinxlineitem{Materials:}\begin{itemize}
\item {} 
\sphinxAtStartPar
\sphinxhref{https://github.com/Elia-Tomaselli/CV-CameraCalibration}{Starting point github}

\item {} 
\sphinxAtStartPar
\sphinxhref{https://drive.google.com/drive/folders/1P6Bs7bx\_CGXWCbx\_5wyAnqc8fPY2SGxO?usp=sharing}{Starting point drive}

\item {} 
\sphinxAtStartPar
\sphinxhref{https://drive.google.com/drive/folders/11RhLrWwb\_tH9uLBCGraR55N0\_Lnnaww-?usp=sharing}{Videos}

\item {} 
\sphinxAtStartPar
\sphinxhref{https://drive.google.com/drive/folders/15\_CCC2mGQZmn3WqdCiWEGJuTSz584Ch0?usp=sharing}{Calibration videos}

\end{itemize}

\sphinxlineitem{Coding:}\begin{itemize}
\item {} 
\sphinxAtStartPar
\sphinxhref{https://docs.opencv.org/4.x/dc/dbb/tutorial\_py\_calibration.html}{Camera Calibration}

\item {} 
\sphinxAtStartPar
\sphinxhref{https://docs.opencv.org/4.10.0/}{OpenCV 4.10 documentation}

\item {} 
\sphinxAtStartPar
\sphinxhref{https://docs.opencv.org/4.10.0/db/d58/group\_\_calib3d\_\_fisheye.html}{OpenCV Fisheye}

\end{itemize}

\sphinxlineitem{Useful links:}\begin{itemize}
\item {} 
\sphinxAtStartPar
\sphinxhref{https://www.fiba.basketball/documents/official-basketball-rules/current.pdf}{FIBA Rules}

\item {} 
\sphinxAtStartPar
\sphinxhref{https://www.fivb.com/wp-content/uploads/2024/03/FIVB-Volleyball\_Rules\_2021\_2024\_pe.pdf}{FIVB Rules}

\end{itemize}

\end{description}


\section{Camera Positions}
\label{\detokenize{resources:camera-positions}}
\noindent{\hspace*{\fill}\sphinxincludegraphics[width=1200\sphinxpxdimen]{{Disposition}.png}\hspace*{\fill}}

\sphinxAtStartPar
Camera positions retrieved from drawing


\begin{savenotes}\sphinxattablestart
\sphinxthistablewithglobalstyle
\raggedright
\begin{tabulary}{\linewidth}[t]{TTTT}
\sphinxtoprule
\sphinxstyletheadfamily 
\sphinxAtStartPar
Cam
&\sphinxstyletheadfamily 
\sphinxAtStartPar
x
&\sphinxstyletheadfamily 
\sphinxAtStartPar
y
&\sphinxstyletheadfamily 
\sphinxAtStartPar
z
\\
\sphinxmidrule
\sphinxtableatstartofbodyhook
\sphinxAtStartPar
1
&
\sphinxAtStartPar
\sphinxhyphen{}15.10
&
\sphinxAtStartPar
\sphinxhyphen{}17.90
&
\sphinxAtStartPar
6.20
\\
\sphinxhline
\sphinxAtStartPar
2
&
\sphinxAtStartPar
0.00
&
\sphinxAtStartPar
\sphinxhyphen{}17.90
&
\sphinxAtStartPar
6.20
\\
\sphinxhline
\sphinxAtStartPar
3
&
\sphinxAtStartPar
\sphinxhyphen{}22.30
&
\sphinxAtStartPar
\sphinxhyphen{}10.20
&
\sphinxAtStartPar
6.60
\\
\sphinxhline
\sphinxAtStartPar
4
&
\sphinxAtStartPar
14.80
&
\sphinxAtStartPar
\sphinxhyphen{}18.10
&
\sphinxAtStartPar
6.20
\\
\sphinxhline
\sphinxAtStartPar
5
&
\sphinxAtStartPar
\sphinxhyphen{}22.00
&
\sphinxAtStartPar
10.20
&
\sphinxAtStartPar
6.80
\\
\sphinxhline
\sphinxAtStartPar
6
&
\sphinxAtStartPar
0.00
&
\sphinxAtStartPar
10.20
&
\sphinxAtStartPar
6.35
\\
\sphinxhline
\sphinxAtStartPar
7
&
\sphinxAtStartPar
24.80
&
\sphinxAtStartPar
0.00
&
\sphinxAtStartPar
6.40
\\
\sphinxhline
\sphinxAtStartPar
8
&
\sphinxAtStartPar
22.00
&
\sphinxAtStartPar
10.00
&
\sphinxAtStartPar
6.35
\\
\sphinxhline
\sphinxAtStartPar
12
&
\sphinxAtStartPar
24.80
&
\sphinxAtStartPar
\sphinxhyphen{}10.00
&
\sphinxAtStartPar
6.90
\\
\sphinxhline
\sphinxAtStartPar
13
&
\sphinxAtStartPar
\sphinxhyphen{}22.00
&
\sphinxAtStartPar
0.00
&
\sphinxAtStartPar
7.05
\\
\sphinxbottomrule
\end{tabulary}
\sphinxtableafterendhook\par
\sphinxattableend\end{savenotes}



\renewcommand{\indexname}{Index}
\printindex
\end{document}